\subsection{Classes}
			
\subsubsection{Action}
This is an action that can be executed in a task of the workflow.
			
\begin{figure}[htbp] \centering
\includegraphics[width=\textwidth]{figures/action.png}
\caption{Action and its relationshiops to other classes in the ontology.}
\label{action}
\end{figure}

\subsubsection{Action Type}
This resource defines the type of the action that will be executed. It
distinguishes among types of actions such as shell functions, user input,
waiting, etc.
			
\begin{figure}[htbp] \centering
\includegraphics[width=\textwidth]{figures/actionType.png}
\caption{Action Type and its subclasses.}
\label{actionType}
\end{figure}
			
\subsubsection{ Basic Action Type } The basic action type is the base class for
basic actions that are typically considered native actions of workflow engines
that execute workflows. This includes actions such as moving files or doing
simple reductions.
\subsubsection{ Boolean Condition } Boolean conditions evaluate Boolean
statements, such as ``if'' statements.
\subsubsection{ Common Workflow Language Tool } This node represents a common
workflow language (CWL) tool, which is a description of a command line tool used
in CWL workflows.
\subsubsection{ Condition } Conditional actions types indicate that the action
described by this type executes for the purpose of evaluating some logical
condition, such as a Boolean statement, a loop, a cycle, or waiting (polling,
checking) for feedback from an external agent.
			
The targetMethod object property of a condition must always point to one of the
functional (i.e. nonconditional) action types.
\subsubsection{ Conditional Action } Conditional actions are actions that
execute conditionally for either conditional tasks (i.e. as part of the
workflow) or as alternative execution flows when the task enters a different
state.

Conditional actions assigned to tasks indicate that the primary action of the
task should be executed according to the conditional action type until the
condition action evaluates to true.
\subsubsection{ Cycle } A cycle describes an action that exits when a condition
describing the end of a cycle has been met. Where a loop action type describes
execution over a range, a cyclic action type checks for the completion of a task
cycle.
\subsubsection{ Generic Executable } The executable action type is the base
class for actions that require executing generic programs on the system.
\subsubsection{ External Agent Condition } The external agent condition
describes an action that waits conditionally on feedback from an external agent,
including a human or an external service. Tasks can block themselves to wait on
feedback, but in some cases an explicit task may exist for a user that can
be described and explicitly executed in the workflow.
\subsubsection{ Fortran Function } The Fortran function action type is the base
class for actions that require executing a function in the Fortran programming
language.
\subsubsection{ Java Class } The Java class action type is the base class for
actions that require executing a class in the Java programming language. Action
targets for this type should point to a single method in the class that will
create all necessary state information and configure the system before
executing. Thus, to execute a class Car, it may make sense to instead call a
builder class such as ``CarBuilder.runCar.''
\subsubsection{ Language Invocation } This action type represents actions to
invocation language-specific calls or executions as part of the workflow. This
could include, for example, executing a method on a native Java class, a
Fortran function or subroutine, or an R function.
\subsubsection{ Loop } The Loop describes an action that exits when a condition
describing the end of a loop has been met. The loop executes over a range and
differs from a cyclic action type because the latter checks for the completion
of a task cycle.
\subsubsection{ Parallel Loop } A parallel loop condition indicates that the
loop may be executed in parallel (i.e. that the iterations of the loop are
independent).
\subsubsection{ Python Script } The Python script action type is the base class
for actions that require executing a script in the Python programming language.
\subsubsection{ RESTful Web Service } This action types describes
Representational State Transfer (RESTful) web services.
\subsubsection{ SOAP Web Service } This action types describes simple object
access protocol (SOAP) web services.
\subsubsection{ Shell Script } The shell script action type is for actions that
require executing shell scripts on systems that support shells.
\subsubsection{ State Change } A state change is executed under the condition
that a task experiences a state change.
\subsubsection{Task}

Tasks are executed by workflows. They are modeled as the combination of an
action and properties defining the way that action should be executed.

Tasks may also be assigned conditional actions that evaluate when a certain
condition has been met based on the execution of the primary action with its
properties.

\begin{figure}[htbp] \centering
\includegraphics[width=\textwidth]{figures/task.png}
\caption{Task and its relationship to actions, state changes, and workflow
descriptions. It is the central element of the workflow ontology.}
\label{task}
\end{figure}


\subsubsection{ WSDL Web Service } This action types describes web service
description language (WSDL) web services.
\subsubsection{ Web Service } The web service action type is the base class for
actions that require executing remote web services.
\subsubsection{Workflow Description}

This class provides a description of the data and tasks that make up a workflow.
It describes a collections of tasks that are executed to accomplish an activity
with certain goals according to various properties and possibly using some data.

\begin{figure}[htbp] \centering
\includegraphics[width=\textwidth]{figures/workflowDescription.png}
\caption{Workflow ontological class showing relationships to tasks and lower
parts of the workflow.}
\label{workflow-description}
\end{figure}


\subsection{Object Properties}
\subsubsection{ depends On } This property indicates that the task (domain)
depends on the successful execution of the range, which is another task or set
of tasks.

It is possible to declare multiple instances of this object property such that
one task will depend on the successful execution of multiple tasks.
\subsubsection{ executes } This property links a workflow description to a task
it should execute.
\subsubsection{ has Action } This object property denotes that the task (domain)
uses the action (range) to which it points.
\subsubsection{ has Action Target } This tag describes the target (program,
function, web service, etc.) the action should execute. Its domain is tied
to action, but its range is open to accommodate whatever the type of the target
is.
\subsubsection{ has Action Type } This property links a concrete action type to
the subject, which must be an action instance.
\subsubsection{ has Condition } This property indicates that the conditional
task (domain) is subject to the completion only if the conditional action
(range) executes successfully.
\subsubsection{ has Properties } This property indicates that the task (domain)
has the properties described by the range. The range is open because the type of
the properties may be undefined.
\subsubsection{ on State Change } This property links a task (domain) to a state
change action (range) it should execute when its state changes.
\subsection{Datatype Properties}
\subsubsection{ has Host } This property describes the host on which a task or
workflow should be executed.
\subsubsection{ has State } This data property describes the present state of
the task or workflow.
\begin{itemize}
  \item \textbf{Initialized} --- This pseudostate indicates that the state
  machine has fully initialized. In practice, full and successful
  initialization results in an immediate local transition to Ready.
  \item \textbf{Failed} --- This state indicates that an unexpected failure
  happened while executing the task.
  \item \textbf{Reviewing} --- The Reviewing state is entered when a task
  needs to spend a large amount of time to review information received for
  pre-, post-, or in situ processing that is required to execute the task.
  Once the review is complete, the task will transition into the Executing
  state in the ideal primary flow.
  \item \textbf{Waiting} --- Tasks in the Waiting state are waiting on
  resources to be properly allocated, including either compute or data
  resources.
  \item \textbf{WaitingForInfo} --- Tasks in the WaitingForInfo state are
  waiting on information from an external agent.
  \item \textbf{Finished} --- This is the terminal state for the task and
  \item \textbf{Executing} --- This state indicates that the task is presently
  executing the work assigned to it.
  \item \textbf{Ready} --- The Ready state indicates that the task can be
  executed and that all initialization has completed or that execution has
  finished and the task is ready to be executed again.
\end{itemize}
\chapter{The Eclipse Integrated Computational Environment} \label{ch:eclipse-ice}

The previous chapter described two assumptions about workflow management
systems, namely that 1) there are no preferred universal languages or tools
and 2) a lack of standardization in software solutions is common because it is
beneficial. Further, that chapter also asserts that an ontological approach
could be used to develop a map of workflow management systems. Those assumptions
and the idea of a scientific workflow ontology arose from nearly a decade of
research into the topic as part of the effort to develop workflow tools for
high-performance modeling and simulation applications. One key realization
during this work was that the system under development could aggregate and share
other workflow engines relatively easily and without significant changes to the
code base, \cite{brooks_introducing_2016}. This suggested that it was possible
to develop a more general, possibly common understanding of workflows and
workflow management systems. That system, the Eclipse Integrated Computational
Environment, is discussed in detail below to introduce concepts that will be
necessary in the development of a scientific workflow ontology in later
chapters.

\baseInclude{pubs/ice-softwarex-2017/src/content}

\section{Summary}

This chapter presents an overview of Eclipse ICE including a discussion on its
position within the broader workflow management system ecosystem and its
workflow model. The overall architecture of Eclipse ICE version 2.0 is described
and several examples are presented to illustrate its applicability to modeling
and simulation problems. Samples and tutorials are also provided for the
eager and interested reader.

\baseInclude{pubs/ice-softwarex-2017/src/conclusions}

The primary challenge with understanding the differences between Eclipse ICE and
these other systems, as well as how they can be integrated, is the lack of a
standard model that can holistically describe multiple types of workflows. The
following chapter looks starts the process of developing such a model by
describe tools and techniques for creating ontologies that describe the entities
and relationships of complex systems.

\chapter{Scientific Workflow Ontology}
\label{ch:workflow-ontology}

The tools, techniques, and background knowledge provided in the previous
chapters makes it possible to answer important questions about eclectic mix of
workflow technologies. These questions include ``What is a workflow?'' and
``Are workflow management systems conceptually the same?,'' all with the goal
of establishing whether the tools are varied or merely variegated. The first of
this two questions - ``What is a workflow?'' - is of particular interest in this
work because a better understanding of the different types of workflows will
present the opportunity to examine problems that were previously too costly in
manual labor or altogether impossible.

The following sections present a workflow ontology, \S
\ref{workflow-ont-section}, and the method used to develop it, \S
\ref{workflow-ont-method}. Finally, conrete examples of the application of this
ontology to workflows ```in the wild'' are provided to show its range as a
decision making and scientific computing tool.

\section{Methodology}
\label{workflow-ont-method}

The scientific workflow ontology presented below was developed by considering
workflows in two environments including 1) the context of problems they solve,
and 2) as entities that are executed by workflow management systems.

\subsection{Workflow Problems}

Workflows are most interesting in the context of problems that they solve. As
chapter \ref{ch:introduction} demonstrates, because of the large number of these
problems that exist, it can be very difficult to write an all-encompassing
definition of a scientific workflow by only looking at the workflows directly. 

One classic way to solve calculus problems without an obvious solution is the
method of change of variables in which new variables related to the original
variables by some relationship are used in place of the originals. Changing
variables makes it possible to mask certain types of complexity to reveal direct
methods of solving the problem. An analogy to this method can be used to study
scientific workflows. Specifically, seeking a definition for workflow problems
instead of workflows can make it possible to find a definition of scientific
workflows by ``solving'' for it. 

It is sufficient for the purpose of this work to define a workflow problem
by building an ontological model based on the description of workflow
management systems, workflows, and data in other chapters. Workflow problems of
any of type can be decomposed into three required components: The workflow
description, the workflow engine or management system that executes the
workflow based on the description, and the data required to fully describe and
execute the workflow. The latter may include - but does not necessarily require
- metadata that describes the contents of the data itself, bulk data including
values and quantities of interest used in the workflow. (For the purposes of
this work, it is sufficient to consider provenance information as a type of
metadata.) A workflow problem, then, is one that is solved by providing a
workflow description to a workflow management system with all pertinent data in
hand. Thus, by examining the set of workflow problems and workflow management
systems, while allowing data to act as a kind of free variable, it is possible
to describe the set of workflows completely.

This is an empirical way of thinking about workflows that results in an emergent
definition, versus a prescribed one. This method accepts that the community will
move as it sees fit, but asserts (quite strongly) that progress can still be
made by considering what exists collectively. The method is also additive since
any new workflow management system can be studied to learn about the
workflows it supports, and the description of those workflows can be added to
the model created by the original effort. As the model grows, it will enclose a
larger area of the workflow space, resulting in the emergence of a new or
updated description of the set of abstract workflows.

This method responds well to a modeling treatment, and, indeed, may be described
as a modling method. All the languages and tools of chapter \ref{ch:ontologies}
can be applied.

\subsection{Referenced Workflow Management Systems}

Workflows from several workflow management systems were examined as part of this
work. This include workflows from Eclipse ICE, Taverna \cite{taverna},
Triquetrum \cite{triquetrum}, Pegasus \cite{pegasus}, the Common Workflow
Language \cite{cwl}, Cylc \cite{cylc}, Chiron \cite{chiron}, Moteur
\cite{Moteur}, and SAW \cite{SAW}. One unnamed heirarchical workflow management
system from Argonne National Laboratory was reviewed as well.

\section{Workflow Ontology}
\label{workflow-ont-section}

This section describes an ontology for scientific workflows created using the
method and philosophy described in the previous section. Class, object
properties, and data properties are listed below. The full OWL ontology,
(created in Prot\'eg\'e), is provided as an TURTL file in Appendix
\ref{app:full-ontology} to preserve space for the narrative here. The full
TURTL file includes some axioms not discussed below. 

Unlike the example in chapter \ref{ch:ontologies}, no individuals are described
in this section. The figures shown below are Graphviz visualizations of
subgraphs of the main ontology graph pulled from Prot\'eg\'e using its OntoGraf
plugin. These figures illustrate the relationships between the core classes and
properties of the ontology. Table \ref{ont-stats-table} summarizes numerous
metrics of the ontology.

\begin{table}[H]
\begin{tabularx}{\textwidth}{|X|X|}
\hline
Triple count & 241 \tabularnewline\hline
Axiom count	& 161	\tabularnewline\hline
Logical axiom count	& 53	\tabularnewline\hline
Declaration axioms count &	40	\tabularnewline\hline
Class count	& 25	\tabularnewline\hline
Object property count	& 9	\tabularnewline\hline
Data property count	& 3	\tabularnewline\hline
Individual count &	0	\tabularnewline\hline
Annotation Property count	& 8 \tabularnewline\hline
SubClassOf	& 21		\tabularnewline\hline
DisjointClasses  &	5 \tabularnewline\hline
SubObjectPropertyOf	& 3 \tabularnewline\hline
ObjectPropertyDomain &	10	\tabularnewline\hline
ObjectPropertyRange	& 7 \tabularnewline\hline
SubDataPropertyOf	& 1 \tabularnewline\hline
DataPropertyDomain	& 4	\tabularnewline\hline
DataPropertyRange &	2 \tabularnewline\hline
AnnotationAssertion	& 68 \tabularnewline\hline
\end{tabularx}
\caption{Ontology Statistics}
\label{ont-stats-table}
\end{table}

The full ontology is also preserved in the Eclipse ICE GitHub repository,
\cite{eclipse-ice-github}.

%%% Include generated ontology text.
\baseInclude{chapters/ice-workflows.out.tex}

\section{Examples and Applications}

\section{Notes}

History:
What was the first appearance of the word workflow and in what context did it
appear?

Background:
Workflows, etc. - Short, then point to longer discussion in the appendix.
Ontologies - Semantic Web, {RDF,RDFS,OWL}, CWL, ICE data structures

Results:
Strategy - Merge CWL and ICE. Develop - if needed - optimization ontology components
Discuss final product. Point to any additional appendices instead of including in the body.

Case Studies:
Data/Analysis - ICEMAN
M\&S - AM?
Optimization - AM or QC?

\section{Writing}

Open question:
*How do cyclic, hierarchical, and multi-facility workflows fit into this?

\textbf{So what do you know?}

I know there are three broad types of scientific workflows. I lump testing
workflows into high-throughput workflows.

Define a workflow problem as any problem that is solved by the execution of a
workflow. I know that a workflow problem of any type can be decomposed into
three required components: the workflow description, the engine that executes
that description, and the data required for the problem. The latter can be
further decomposed into metadata, bulk data, and provenance.

A workflow may be defined as a collection of tasks that are executed in some
order by human and non-human actors. A workflow problem can then be defined as
any problem that is solved the the execution of a specific workflow. Many
systems exist that can execute workflows encoded in one or more
\textit{description formats} for both business and scientific problems. There
are predominantly three types of scientific workflows: High-throughput \cite{},
Modeling and Simulation \cite{}, and Analysis \cite{}.

Workflow problems of any of type can be decomposed into three required
components: The workflow description, the \textit{workflow engine} that
executes the workflow based on the description, and the data required to fully
describe and execute the workflow. The latter may include - but does not
necessarily require - metadata that describes the contents of the data itself,
bulk data including values and quantities of interest used in the workflow.
(For the purposes of this work, it is sufficient to consider provenance
information as a type of metadata.)

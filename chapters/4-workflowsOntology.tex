\chapter{Scientific Workflow Ontology}

\baseInclude{chapters/ice-workflows.out.tex}


A workflow may be defined as a collection of tasks that are executed in some
order by human and non-human actors. A workflow problem can then be defined as
any problem that is solved the the execution of a specific workflow. Many
systems exist that can execute workflows encoded in one or more
\textit{description formats} for both business and scientific problems. There
are predominantly three types of scientific workflows: High-throughput \cite{},
Modeling and Simulation \cite{}, and Analysis \cite{}.

Workflow problems of any of type can be decomposed into three required
components: The workflow description, the \textit{workflow engine} that
executes the workflow based on the description, and the data required to fully
describe and execute the workflow. The latter may include - but does not
necessarily require - metadata that describes the contents of the data itself,
bulk data including values and quantities of interest used in the workflow.
(For the purposes of this work, it is sufficient to consider provenance
information as a type of metadata.)

\section{Notes}

History:
What was the first appearance of the word workflow and in what context did it
appear?

Background:
Workflows, etc. - Short, then point to longer discussion in the appendix.
Ontologies - Semantic Web, {RDF,RDFS,OWL}, CWL, ICE data structures

Results:
Strategy - Merge CWL and ICE. Develop - if needed - optimization ontology components
Discuss final product. Point to any additional appendices instead of including in the body.

Case Studies:
Data/Analysis - ICEMAN
M\&S - AM?
Optimization - AM or QC?

\section{Writing}

Open question:
*How do cyclic, hierarchical, and multi-facility workflows fit into this?

\textbf{So what do you know?}

I know there are three broad types of scientific workflows. I lump testing
workflows into high-throughput workflows.

Define a workflow problem as any problem that is solved by the execution of a
workflow. I know that a workflow problem of any type can be decomposed into
three required components: the workflow description, the engine that executes
that description, and the data required for the problem. The latter can be
further decomposed into metadata, bulk data, and provenance.

A workflow may be defined as a collection of tasks that are executed in some
order by human and non-human actors. A workflow problem can then be defined as
any problem that is solved the the execution of a specific workflow. Many
systems exist that can execute workflows encoded in one or more
\textit{description formats} for both business and scientific problems. There
are predominantly three types of scientific workflows: High-throughput \cite{},
Modeling and Simulation \cite{}, and Analysis \cite{}.

Workflow problems of any of type can be decomposed into three required
components: The workflow description, the \textit{workflow engine} that
executes the workflow based on the description, and the data required to fully
describe and execute the workflow. The latter may include - but does not
necessarily require - metadata that describes the contents of the data itself,
bulk data including values and quantities of interest used in the workflow.
(For the purposes of this work, it is sufficient to consider provenance
information as a type of metadata.)

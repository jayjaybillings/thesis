\chapter{Scientific Workflow Ontology}
\label{ch:workflow-ontology}

The tools, techniques, and background knowledge provided in the previous
chapters makes it possible to answer important questions about an eclectic mix
of workflow technologies. These questions include ``What is a workflow?'' and
``Are workflow management systems conceptually the same?,'' ---all with the goal
of establishing whether the tools are varied or merely variegated. The first of
these two questions ---``What is a workflow?''--- is of particular interest in
this work because a better understanding of the different types of workflows will
present the opportunity to examine problems that were previously too costly in
manual labor or altogether impossible.

The following sections present a workflow ontology, \S
\ref{workflow-ont-section}, and the method used to develop it, \S
\ref{workflow-ont-method}. Finally, concrete examples of the application of this
ontology to workflows ```in the wild'' are provided to show its range as a
decision-making and scientific computing tool.

\section{Methodology}
\label{workflow-ont-method}

\todo{Talk about design philosophy?}

The scientific workflow ontology presented in \S \ref{workflow-ont-section} was
developed by considering workflows in two environments: i) the context
of problems they solve and ii) as entities that are executed by workflow
management systems.

\subsection{Workflow Problems}

Workflows are most interesting in the context of problems they solve. As
Chapter \ref{ch:introduction} demonstrates, because of the large number of these
problems, it can be very difficult to write an all-encompassing
definition of a scientific workflow by looking only at the workflows directly. 

One classic way to solve calculus problems without an obvious solution is the
method of change of variables in which new variables related to the original
variables by some relationship are used in place of the originals. Changing
variables makes it possible to mask certain types of complexity to reveal direct
methods of solving the problem. An analogy to this method can be used to study
scientific workflows. Specifically, seeking a definition for workflow problems
instead of workflows can make it possible to find a definition of scientific
workflows by ``solving'' for it. 

It is sufficient for the purpose of this work to define a workflow problem
by building an ontological model based on the description of workflow
management systems, workflows, and data in other chapters. Workflow problems of
any of type can be decomposed into three required components: i) The workflow
description, ii) the workflow engine or management system that executes the
workflow based on the description, and iii) the data required to fully describe
and execute the workflow. The latter may include ---but does not necessarily
require--- metadata that describes the contents of the data itself, bulk data
including values and quantities of interest used in the workflow. (For the purposes of
this work, it is sufficient to consider provenance information as a type of
metadata.) A workflow problem, then, is one that is solved by providing a
workflow description to a workflow management system with all pertinent data in
hand. Thus, by examining the set of workflow problems and workflow management
systems, while allowing data to act as a kind of free variable, it is possible
to describe the set of workflows completely.

This is an empirical way of thinking about workflows that results in an emergent
definition, versus a prescribed one. This method accepts the community will
move as it sees fit, but asserts (quite strongly) that progress can still be
made by considering what exists collectively. The method is additive since
any new workflow management system can be studied to learn about the
workflows it supports, and the description of those workflows can be added to
the model created by the original effort. As the model grows, it will enclose a
larger area of the workflow space, resulting in the emergence of a new or
updated description of the set of abstract workflows.

This method responds well to a modeling treatment, and, indeed, may be described
as a modeling method. All the languages and tools in Chapter \ref{ch:ontologies}
can be applied.

\subsection{Referenced Workflow Management Systems}

Workflows from several workflow management systems were examined as part of this
work. This include workflows from Eclipse ICE, Taverna \cite{taverna},
Triquetrum \cite{triquetrum}, Pegasus \cite{pegasus}, the Common Workflow
Language \cite{cwl}, Cylc \cite{cylc}, Chiron \cite{chiron}, Moteur
\cite{Moteur}, and SAW \cite{SAW}. One unnamed heirarchical workflow management
system from Argonne National Laboratory was also reviewed as well.

\section{Workflow Ontology}
\label{workflow-ont-section}

This section describes an ontology for scientific workflows created using the
method and philosophy described in the previous section. Classes, object
properties, and data properties are listed subsequently. The full OWL ontology,
(created in Prot\'eg\'e), is provided as a TURTL file in Appendix
\ref{app:full-ontology} to preserve space for the narrative here. The full
TURTL file includes some axioms not discussed here. 

Unlike the example in Chapter \ref{ch:ontologies}, no individuals are described
in this section. The following figures are Graphviz visualizations
of subgraphs of the main ontology graph pulled from Prot\'eg\'e using its OntoGraf
plugin. These figures illustrate the relationships among the core classes and
properties of the ontology. Table \ref{ont-stats-table} summarizes numerous
metrics of the ontology.

\begin{table}[H]
\begin{tabularx}{\textwidth}{|X|X|}
\hline
Triple count & 241 \tabularnewline\hline
Axiom count	& 161	\tabularnewline\hline
Logical axiom count	& 53	\tabularnewline\hline
Declaration axioms count &	40	\tabularnewline\hline
Class count	& 25	\tabularnewline\hline
Object property count	& 9	\tabularnewline\hline
Data property count	& 3	\tabularnewline\hline
Individual count &	0	\tabularnewline\hline
Annotation property count	& 8 \tabularnewline\hline
SubClassOf	& 21		\tabularnewline\hline
DisjointClasses  &	5 \tabularnewline\hline
SubObjectPropertyOf	& 3 \tabularnewline\hline
ObjectPropertyDomain &	10	\tabularnewline\hline
ObjectPropertyRange	& 7 \tabularnewline\hline
SubDataPropertyOf	& 1 \tabularnewline\hline
DataPropertyDomain	& 4	\tabularnewline\hline
DataPropertyRange &	2 \tabularnewline\hline
AnnotationAssertion	& 68 \tabularnewline\hline
\end{tabularx}
\caption{Ontology Statistics}
\label{ont-stats-table}
\end{table}

The full ontology is also preserved in the Eclipse ICE GitHub repository
\cite{eclipse-ice-github}.

%%% Include generated ontology text.
\baseInclude{chapters/ice-workflows.out.tex}

\section{Examples and Applications}
\label{workflow-ont-examples}

This section shows examples and applications of workflows marked up as
instances of the workflow ontology. The full RDF listings for all
examples are provided in the appendices.

\subsection{Basic File Move}
\label{move-workflow}

The first example is a very simple workflow showing the move of a file from
one location to another. The workflow description executes a single task, which
is specialized to move a file using a dedicated action type. The action type
uses a Java class as its target, and it uses simple string values for file name
input and output.

This example is important because it shows how simple it is to wire together a
straightforward workflow. The relationships among workflow descriptions,
tasks, and actions create a directed acyclic graph for this type of workflow.
However, it is also important because file transfer is a common task in many
workflows, often occurring as a subworkflow that executes before and after
other tasks or before and after the main workflow.

Figure \ref{move-workflows} shows a version of this example that has been
slightly edited to better fit the page. The full TURTL version of this workflow
is available in the appendices, \S \ref{app:move-workflow}.

\begin{figure}[htbp]
\centering
\baseIncludegraphics{figures/moveWorkflow.png}
\caption{The basic file move workflow.}
\label{move-workflows}
\end{figure}

\subsection{Combining Cycles and Loops}

The next example covers a more complicated use case that is arguably uncommon
in more popular workflow engines: cycles and loops. This example, depicted in
figure \ref{cycle-loop-test}, describes a simple workflow were 50 files are
created using a threshold limiter and then deleted through a standard
50 iteration loop. A threshold limiter limits the amount of something --- in
this case the number of files created --- based on a threshold.

This workflow is a simple model, but it is a good analog to more complicated
systems in which data is gathered until a threshold is crossed, such as
the number of counts from a detector, and then the set of files are processed in a loop. A
threshold is well modeled by a cycle because the system waits until the
threshold is met, which requires a periodic (cyclic) check against the limit.

The examples contains two tasks, \#loopTask and \#cycleTask, with the \#loopTask
being dependent on \#cycleTask. The dependency is because \#cycleTask
creates the files, which must all exist before the \#loopTask is executed. The
action of the cycle task is to create a file using the Linux ``touch'' command,
and its condition is to created files until the number of files in the directory
is greater than 50. It checks the number of files using a Java program called
``fileCounter.'' When this task is complete, the dependency for \#loopTask is
satisfied and it can execute its action --- the Linux ``rm'' command to remove a
file. The condition on the loop is that its action is executed for fifty
iterations specified by the lowerBound, upperBound, and stepSize properties.

Workflows such as this are easy to model in workflow systems where loops and
cycles are supported. It is also possible to execute this workflow in systems
that do not directly support those constructs by decomposing it into smaller
workflows since the number of files is fixed. The \#cycleTask can be
executed as a linear graph of fifty separate system checks. The \#loopTask can
be unrolled to fifty separate executions of the remove command. However, limited
polling and unrolling, respectively, only work in cases where the total number
of iterations is fixed.

\begin{figure}[htbp]
\centering
\baseIncludegraphics{figures/cycle-loop-test.png}
\caption{Combined cycle and loop workflow example.}
\label{cycle-loop-test}
\end{figure}

The full TURTL version of this workflow is available in the appendices, \S
\ref{app:cycle-loop-test}.

\subsection{Pegasus Split Example}

The Pegasus workflow management system is especially good at executing large,
parallel workflows. The Pegasus website and documentation provide a simple
example of splitting files into parts in parallel \cite{}. The workflow model
of Pegasus is very well paired with the workflow ontology, and mapping this
example from its source (\cite{}) can be accomplished in a very straightforward
fashion.

Figure \ref{pegasus-split-workflow} is graph of the Pegasus splitting example
showing only the most essential parts, namely, tasks, actions, action types, and
properties from the workflow ontology. Files and parameters in Pegasus map
directly to properties in the ontology, with the jobs that are executed against
them mapping to tasks and actions. This example illustrates how quickly even a
simple workflow can become too hard to easily visualize in its entire scope, so
figure \ref{pegasus-comparison} shows i) a greatly slimmed down version of the
worfklow graph that contains only the tasks compared against ii) the original
graph of the image from the Pegasus website.

One important difference of the workflow ontology that is highlighted by
Figure \ref{pegasus-comparison} is that it models workflow tasks and their
dependencies, whereas other workflow models are focused on data flow. This is
highlighted by the inversion of the arrow heads across the sides of the image.
In part a), the arrows point from the tasks at the top of the dependency chain
down to the initial task that has no dependencies against it. The opposite is
true in part b), which shows the initial task with no dependencies generating
output that is fed into the tasks where it is required. 

Both views are correct: They are equally valid perspectives are akin
to saying that ``The donkey pulls the cart'' and ``The cart is pulled by the
donkey.'' As long as the interpreter gets the point that the cart needs to move,
there is not problem. This inverted graph phenomenon is also witnessed in
provenance graphs when compared against workflows they describe (see \S
\ref{provenance-background}).

\begin{figure}[htbp]
\centering
\baseIncludegraphics{figures/pegasusSplit.png}
\caption{Only the most essential elements of the Pegasus split
example as marked up in the workflow ontology. This level of detail ---far
removed from the full details--- shows the high number of facts captured with
semantic models.}
\label{pegasus-split-workflow}
\end{figure}

\begin{figure}[htbp]
\centering
\baseIncludegraphics{figures/pegasus-comparison.png}
\caption{Graph of Figure \ref{pegasus-split-workflow} on the left with only
the tasks shown compared with the original Pegasus split example from the
website on the right.}
\label{pegasus-comparison}
\end{figure}

The full TURTL version of this workflow is available in the appendices, \S
\ref{app:pegasus-workflow}.

\subsection{Eclipse ICE II/III Task Model}

Eclipse ICE, covered extensively in Chapter \ref{ch:eclipse-ice},
uses a somewhat unique view of workflows in its workflow model. Workflows
in Eclipse ICE are executed as finite state machines that can include human
feedback, conditional branching, and error conditions. This makes it possible
to describe workflows in a conceptually abstract way using state machine
theory. It remains possible to model the execution flow of these state machines
as directed acyclic graphs, which can be observed in Figure \ref{ice-workflow}.
When cast into the workflow ontology, it is clear that the abstract workflow
executed by ICE is directed and acyclic, even if its instances may execute with
cycles and other conditions.

ICE workflows start by setting up the Form used to collect
\#workflowProperties, which is the \#setupFormTask. The workflow continues
through multiple tasks that depend on the initial Form, as well as other tasks,
and which cause state changes within the system. On state changes, the task
executes the \#stateChangeAction to update the system state, reconfigure data,
and prepare the next task. User feedback is required, and the \#submitForm task
will wait until this condition is met before transitioning to review and
processing. Like the earlier examples, arrows point from tasks to previously
executed tasks through the ``dependsOn'' property such that the last task,
\#processTask, shows up at the top and does not appear to feed any other tasks.

The system goes from one task to another through the state changes. Each state
change configures the system so it can execute its next task and ensures
that all previous steps were executed properly. Thus, the internal logic of when
a task is complete remains internal to the workflow, and no external logic is
required to satisfy task completion.

This example shows a very interesting relationship between the \#hasCondition
and \#dependsOn object properties in the ontology: Conditions are merely
special, internal dependencies that are managed directly by the task instead of the
workflow engine. Conditions execute small ``micro-workflows'' within the task
that affect its completion, while the workflow engine manages its completion by
first executing the tasks on which it depends. Thus, conditions can be thought
of as merely subgraphs of the larger workflow that sit between a task and its
dependencies.

\begin{figure}[htbp]
\centering
\baseIncludegraphics{figures/iceWorkflow.png}
\caption{The tasks, actions, and states of the standard workflow model of
Eclipse ICE, as graphed using the workflow ontology.}
\label{ice-workflow}
\end{figure}

The full TURTL version of this workflow is available in the appendices, \S
\ref{app:ice-workflow}.

\subsection{Neutron Scattering User Workflow and Data Pipeline}

The Spallation Neutron Source and the High-Flux Isotope Reactor operated by Oak
Ridge National Laboratory are the premiere neutron sources in the United States.
Figure \ref{neutron-workflow} shows the idealized, conceptual workflow that
users can expect while performing experiments at the facility
\cite{sns-hfir-data-review}. This is a good example of the conceptual
workflows mentioned in Chapter \ref{ch:introduction}, \S \ref{workflows},
because it cannot be executed and represents high-level, idealized tasks.

\begin{figure}[htbp]
\centering
\baseIncludegraphics{figures/neutronWorkflow.png}
\caption{Conceptual neutron scattering user workflow and data pipeline.}
\label{neutron-workflow}
\end{figure}

Figure \ref{neutron-workflow-graph} shows the graph of this workflow when
translated to the workflow ontology. One immediate observation is that
conceptual workflows may not require a deep translation to be understandable
and that in some cases only tasks are needed to describe conceptual workflows.
Still, it is useful to look at conceptual worfklows in the same framework as
concrete workflows to understand how conceptual workflows can evolve to be
concrete and executable in the future.

Another important observation is that this workflow is cyclic. It is tempting to
believe that the Design of Experiments task is the first task because it is the
top-most, left-most task in the diagram, which to English speakers may suggest
that it is special. It is true that in some cases Design of Experiments may be
the first task, but many users walking through this workflow start at other
points, such as ``HPC, Modeling/Simulation, AI/ML'' because they approach
the problem from a theoretical perspective. There are several others cycles
between the various parts of the graph, such as the ``Data Reduction'' to
``Data Curation and Archival'' to ``HPC, Modeling/Simulation, AI/ML'' cycle.

\begin{figure}[htbp]
\centering
\baseIncludegraphics{figures/neutronWorkflow-graph.png}
\caption{Conceptual neutron scattering user workflow and data pipeline
graphed in the workflow ontology.}
\label{neutron-workflow-graph}
\end{figure}

The full TURTL version of this workflow is available in the appendices, \S
\ref{app:neutron-workflow}.

\section{Summary}
\label{workflows-ont-summary}

This chapter presents the methodology and reasoning behind creating a workflow
ontology using the methods discussed in Chapter \ref{ch:ontologies}, the
ontology for scientific workflows itself, and five examples of scientific
workflows mapped or translated into instance graphs of the ontology. The
examples, in particular, demonstrate the central thesis of this work that a
comprehensive metamodel of scientific workflows could describe multiple types of
scientific workflows. This includes high-throughput (Pegasus); modeling and
simulation (Eclipse ICE); iterative workflows that run repeated, looping, and
cyclic tasks (the cyclic looping example); and purely conceptual workflows that
are not executed by machines; but remain important for communication and other
work (neutron scattering user workflow and data pipeline). The examples
further demonstrate that common patterns can be described by the ontology
in a common way that is agnostic to the underlying platform, much like
design patterns in programming (Move Workflow example). 

What remains is the final question of this thesis: If scientific workflows can
be uniformly described, what are the implications for interoperability? Chapter
\ref{ch:interoperability} will seek to address this question by examining the
many different facets of interoperability.

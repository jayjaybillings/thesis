%%%%%%%%%%%%%%%%%%%%%%%%%%%%%%%%%%%%%%%%%%%%%%%%%%%%%%%%%%%%%%%%%%%%%%%%%%%%%%%%%%%%%%%%%%%%%%%%%%%%%
% This template is distributed with ABSOLUTELY NO WARRANTY.
% It serves as a guideline and constitutes a basic structure for a
% thesis/dissertation. The user assumes full responsibility for formatting
% and typesetting their document and for verifying that all the thesis
% requirements set by the University of Tennessee are met. Please refer to the most
% recent UT thesis guide (http://gradschool.utk.edu/thesesdissertations/formatting/)
% or contact the thesis consultant (http://gradschool.utk.edu/thesesdissertations/).
% Please report any bugs to the thesis consultant.
%%%%%%%%%%%%%%%%%%%%%%%%%%%%%%%%%%%%%%%%%%%%%%%%%%%%%%%%%%%%%%%%%%%%%%%%%%%%%%%%%%%%%%%%%%%%%%%%%%%%%
% O P T I O N S:
% 1. thesis/dissertation
% 2. monochrome
% 3. all options provided by the report class
%%%%%%%%%%%%%%%%%%%%%%%%%%%%%%%%%%%%%%%%%%%%%%%%%%%%%%%%%%%%%%%%%%%%%%%%%%%%%%%%%%%%%%%%%%%%%%%%%%%%%
%First, is this a thesis or dissertation? Choose one by commenting out the one you don't need:
%\documentclass[thesis,letterpaper,12pt]{utthesis} % thesis
\documentclass[dissertation,letterpaper,12pt]{utthesis} %dissertation
% some alternatives are:
%\documentclass[thesis,monochrome,letterpaper,12pt]{utthesis} %thesis, monochrome text
\renewcommand{\baselinestretch}{1.5} 	 % line Spacing
%%%%%%%%%%%%%%%%%%%%%%%%%%%%%%%%%%%%%%%%%%%%%%%%%%%%%%%%%%%%%%%%%%%%%%%%%%%%%%%%%%%%%%%%%%%%%%%%%%%%%
% TO DO: FILL IN YOUR INFORMATION BELOW - READ THIS SECTION CAREFULLY
%%%%%%%%%%%%%%%%%%%%%%%%%%%%%%%%%%%%%%%%%%%%%%%%%%%%%%%%%%%%%%%%%%%%%%%%%%%%%%%%%%%%%%%%%%%%%%%%%%%%%
\title{Ontological Considerations for Interoperability in Scientific Workflows}	       	% title of thesis/dissertation
\author{Jay Jay Billings}                			% author's name
\copyrightYear{20xx}            				% copyright year of your thesis/dissertation
\graduationMonth{YYY}           				% month of graduation for your thesis/dissertation
\degree{Doctor of Philosophy}	    			% degree: Doctor of Philosophy, Master of Science, Master of Engineering...
\university{The University  of Tennessee, Knoxville}	% school name
%%%%%%%%%%%%%%%%%%%%%%%%%%%%%%%%%%%%%%%%%%%%%%%%%%%%%%%%%%%%%%%%%%%%%%%%%%%%%%%%%%%%%%%%%%%%%%%%%%%%%
% LOAD SOME USEFUL PACKAGES. 
% No need to change anything here, although if you'd like to add packages you can do that here. Note that packages preloaded with the utthesis class are: amsmath,amsthm,amssymb,setspace,geometry,hyperref,and color
%%%%%%%%%%%%%%%%%%%%%%%%%%%%%%%%%%%%%%%%%%%%%%%%%%%%%%%%%%%%%%%%%%%%%%%%%%%%%%%%%%%%%%%%%%%%%%%%%%%%%
\usepackage{nomencl}                    % produces a nomenclature
\usepackage{float}                      % figure floats
\usepackage[numbers]{natbib}                     % this package allows you to link your references
\usepackage{graphicx}					% graphics package
\graphicspath{ {figures/}{figures/eps/}{figures/pdf/} }% specify the path where figures are located
\usepackage{fancyhdr}                   % fancy headers and footers
\usepackage{url}                        % nicely format url breaks
\usepackage[inactive]{srcltx}		 	% necessary to use forward and inverse searching in DVI
\usepackage{relsize}                    % font sizing hierarchy
\usepackage{booktabs}                   % professional looking tables
\usepackage[config, labelfont={bf}]{caption,subfig} % nice sub figures
\usepackage{mathrsfs}                   % additional math scripts
\usepackage[titletoc]{appendix}			% format appendix correctly
\usepackage{pdflscape}					% to produce landscape pages if necessary
\usepackage{thesistools}				% my thesis tools package for cramming all these papers together
\usepackage{tabularx}					% tabularx package for better tables
\usepackage{todonotes}					% this package captures notes and tasks TO-DO
\usepackage[autostyle]{csquotes} 		% stylish quote package
\usepackage{array}						% array-listing package
\usepackage{tabu}						% alternate table package
\usepackage{listings}					% listings package for code listings, etc.
\usepackage{color}						% Color package for working with the listing package

%%%%%%%%%%%%%%%%%%%%%%%%%%%%%%%%%%%%%%%%%%%%%%%%%%%%%%%%%%%%%%%%%%%%%%%%%%%%%%%%%%%%%%%%%%%%%%%%%%%%%%%
%
% lstlisting settings for syntax highlighting
%
%%%%%%%%%%%%%%%%%%%%%%%%%%%%%%%%%%%%%%%%%%%%%%%%%%%%%%%%%%%%%%%%%%%%%%%%%%%%%%%%%%%%%%%%%%%%%%%%%%%%%%%

% Color definitions
\definecolor{mygreen}{rgb}{0,0.6,0}
\definecolor{mygray}{rgb}{0.5,0.5,0.5}
\definecolor{mymauve}{rgb}{0.58,0,0.82}
\definecolor{maroon}{rgb}{0.5,0,0}
\definecolor{darkgreen}{rgb}{0,0.5,0}

% Basic setup
\lstset{ %
  backgroundcolor=\color{white},   % choose the background color
  basicstyle=\footnotesize,        % size of fonts used for the code
  breaklines=true,                 % automatic line breaking
  breakatwhitespace=false,         % allow breaking in character space
  linewidth=\textwidth,			   % make sure the listings are \textwidth wide
  captionpos=tb,                    % sets the caption-position to top
  commentstyle=\color{mygreen},    % comment style
  escapeinside={\%*}{*)},          % if you want to add LaTeX within your code
  keywordstyle=\color{blue},       % keyword style
  stringstyle=\color{mymauve},     % string literal style
}

% XML redefinition
\lstdefinelanguage{XML} {
  basicstyle=\ttfamily,
  morestring=[s]{"}{"},
  morecomment=[s]{?}{?},
  morecomment=[s]{!--}{--},
  commentstyle=\color{darkgreen},
  moredelim=[s][\color{black}]{>}{<},
  moredelim=[s][\color{red}]{\ }{=},
  stringstyle=\color{blue},
  identifierstyle=\color{maroon}
}

% TURTLE definition
\lstdefinelanguage{TURTL} {
  basicstyle=\ttfamily,
%  morestring=[s]{"}{"},
%  morecomment=[s]{?}{?},
%  morecomment=[s]{\#\#\#}{\\n},
%  commentstyle=\color{darkgreen},
  %moredelim=[s][\color{red}]{@prefi}{x},
  %moredelim=[s][\color{blue}]{rdf}{:},
%  stringstyle=\color{blue},
  identifierstyle=\color{maroon}
}

%%%%%%%%%%%%%%%%%%%%%%%%%%%%%%%%%%%%%%%%%%%%%%%%%%%%%%%%%%%%%%%%%%%%%%%%%%%%%%%%%%%%%%%%%%%%%%%%%%%%%%
% This section formats landscape pages properly with the correct page number.
% This code is only necessary when landscape pages are needed and can be left alone
%%%%%%%%%%%%%%%%%%%%%%%%%%%%%%%%%%%%%%%%%%%%%%%%%%%%%%%%%%%%%%%%%%%%%%%%%%%%%%%%%%%%%%%%%%%%%%%%%%%%%%

\fancypagestyle{mylandscape}{
	\fancyhf{} %Clears the header/footer
	\fancyfoot{% Footer
    \makebox[\textwidth][r]{% Right
      \rlap{\hspace{.75cm}% Push out of margin by \footskip
        \smash{% Remove vertical height
          \raisebox{4.87in}{% Raise vertically
            \rotatebox{90}{\thepage}}}}}}% Rotate counter-clockwise
  \renewcommand{\headrulewidth}{0pt}% No header rule
  \renewcommand{\footrulewidth}{0pt}% No footer rule
}


%%%%%%%%%%%%%%%%%%%%%%%%%%%%%%%%%%%%%%%%%%%%%%%%%%%%%%%%%%%%%%%%%%%%%%%%%%%%%%%%%%%%%%%%%%%%%%%%%%%%%
\begin{document}
    \pagenumbering{alph} % this is needed to clear certain issues with the hyperref package
    %
    \addToPDFBookmarks{0}{Front Matter}{rootNode} % create a root node named "Front Matter" in the pdf bookmarks
    \addToPDFBookmarks{1}{Title}{a} % add a pdf bookmark to the title page
    \makeTitlePage % make the title page.
    %
    \pagenumbering{roman}
    \setcounter{page}{2}
    %
    \makeCopyrightPage % make the copyright page
    %
%%%%%%%%%%%%%%%%%%%%%%%%%%%%%%%%%%%%%%%%%%%%%%%%%%%%%%%%%%%%%%%%%%%%%%%%%%%%%%%%%%%%%%%%%%%%%%%%%%%%%
%The dedication and acknowledgments are optional. If you wish not to include them, simply comment out both the "\addToPDF..." line and the "\include{...}" line for each.
%%%%%%%%%%%%%%%%%%%%%%%%%%%%%%%%%%%%%%%%%%%%%%%%%%%%%%%%%%%%%%%%%%%%%%%%%%%%%%%%%%%%%%%%%%%%%%%%%%%%%
    \addToPDFBookmarks{1}{Dedication}{b} % add a pdf bookmark to the dedication page
    \chapter*{}
\begin{center}
{\centering \it dedication... }
\end{center}  % include the dedication

    \addToPDFBookmarks{1}{Acknowledgments}{c} % add a pdf bookmark to the acknowledgments page
    \chapter*{Acknowledgments}

There is an old saying among parents: It takes a village to raise a child. The
same phrase likely applies to the development of any dissertation, and as such
the author would like to acknowledge the contributions and support of the
following fellow villagers.

First, the author's committee requires special acknowledgment and gratitude
for taking on an nontraditional student and an interdisciplinary topic.
Neither are easy and they did both at the same time. The committee includes Jack
Dongarra, John Drake, Mike Guidry (chair), Mallikarjun Shankar, and John Turner. 

The author is grateful for the support provided by the US Department
of Energy and the ORNL Director’s Research and Development Fund in the
Integrated Computational Environment for the Modeling and Analysis of Neutrons
(ICEMAN) project. ORNL is managed by UT-Battelle LLC for the US Department of
Energy under contract DE-AC05-00OR22725. The completion of this work would not
have been possible without the continued encouragement and uplifting support of
colleagues and management at ORNL. The author is also grateful to the management
and staff of RNET Technologies LLC, especially Gerald Sabin and Ben O'Neill, who
supported part of this work under subcontract to ORNL.

The author would like to express his sincere thanks to his coauthors and
fellow developers of the Eclipse Integrated Computational Environment that
forms much of the foundation of this work. These collaborators include Andrew
Bennett, Jordan Deyton, Kasper Gammeltoft, Jonah, Graham, Dasha Gorin, Hari
Krishnan, Menghan Li, Alexander J. McCaskey, Taylor Patterson, Robert Smith,
Gregory Watson, and Anna Wojtowicz. Other collaborators who the author would
like to acknowledge for thoughtful discussions on workflows include Jim Belak on
the nature of workflows in the ExAM project, and Robert Clay, Dan Laney, and
David Montoya on modeling and simulation workflows. The earliest work on this
project included substantial effort by many others who directly or
indirectly contributed to the project, either in its early days as “NiCE” or
once it moved to Eclipse. This includes Ronald Allen, Andrew Belt, David E.
Bernholdt, Tim Bohn, Erica Grant, John M. Hetrick III, Forest Hull, Sebastien
Jourdain, JiSoo Kim, Allison Koenecke, Fangzhou Lin, Eric J. Lingerfelt, Greg
Lyon, Tony McCrary, Elizabeth Piersall, Neeti Pokhriyal, Adrian Sanchez, Claire
Saunders, Nick Stanish, Matthew Wang, ivand Scott Wittenberg. The authors would
like to acknowledge the special contribution of the Eclipse Foundation, the
Eclipse Community, the Eclipse Science Working Group, and our many colleagues
who use and contribute to open-source projects in the Eclipse ecosystem.
Finally, the development team is especially grateful to Barney Maccabe, David
Pointer for their endless support and advocacy for this work.

The author is grateful for the information and citations on Scrybe from Anthony
Skjellum, Director of the SimCenter at the University of Tennessee,
Chattanooga. The author is also grateful to the internal reviewers and helpful
colleagues at the Oak Ridge National Laboratory, and many other anonymous
reviewers who provided feedback on the journal articles and conference
presentations that sprung from this work.

Shantenu Jha of Rutgers and Brookhaven National Laboratory, and Katie Knight, of
Oak Ridge National Laboratory, deserve their own acknowledgements chapter and
together could easily serve as the model for an entire ontology for ``Ideal
Colleagues.'' The kindness, encouragment, and continued support, as well as the
constant willingness to act as a sounding board for crazy ideas has earned these
perpetual thanks and beer at the authors expense.

Charlie Horak edited this work, including the author's bad references to Star
Trek in Chapter \ref{ch:ontologies}, and the author is grateful for her patience
dealing with inconsistent commas, tense switching, bad numbering, and a number
of other things that are not in the version the reader is holding thanks to her
efforts.

An extremely large amount of peanut butter and chocolate was consumed in the
creation of this work. This includes a very large amount of the Peanut Butter
Fudge ice cream available at the Island Scoop Restaurant in Jamestown Rhode
Island. The author appreciates the kindness of the proprietor(s) and staff who
let him crash at the restaurant for several hours a day while he worked on
corrections during his family vacation.

Last but not least, the author would like to acknowledge the sacrifies made by
and continuous support and love from his wife and daughter, both of whom
constantly encouraged him, even on vacation, to complete this work. Of all the
villagers who made this possible, none did more than these two and the author
expresses his extreme gratitude and love to them.
 % include the acknowledgments
    
    \addToPDFBookmarks{1}{Abstract}{e} % add a pdf bookmark to the abstract page
    \chapter*{Abstract}\label{ch:abstract}

Scientific workflows exist in many different domains and for many different
computing platforms. As these systems have proliferated, they have also become
increasingly complex and harder to maintain. Furthermore, these systems often
exist as self-sufficient islands of capability that can be over-specialized and
locked into a specific domain. Some commonality exists and three major workflow
types are readily apparent in (i) modeling and simulation, (ii) high-throughput
data analysis, and (iii) optimization. A far more detailed understanding of
different workflow types is required to determine how large, interdisciplinary
workflows that span the types and multiple computing facilities can be created
and executed. This work presents a new model of scientific workflows that
attempts to create such an understanding with a formal, machine-readable
ontology that can be used to answer design questions about interoperability for
workflows that need to be executed across distributed workflow management
systems. Example instances are presented for simple workflows that do not
require decision making, more complicated workflows that can split decision
making between external agents and internal state transitions in finite state
machines, and purely conceptual workflows that represent notional if not exactly
executable workflows purely for communicating ideas. Finally, a perspective on
interoperability for workflow systems is presented in the context of the
ontology. % your abstract

    \addToPDFBookmarks{0}{Table of Contents}{f}
    \tableofcontents % generate a table of contents
    \listoftables % generate a list of tables
    \listoffigures % generate a list of figures
   
    \newpage
    \pagenumbering{arabic}
    \setcounter{page}{1}
    %%%%%%%%%%%%%%%%%%%%%%%%%%%%%%%%%%%%%%%%%%%%%%%%%%%%%%%%%%%%%%%%%%%%%%%%%%%%%%%%%%%%%%%%%%%%%%%%%%%%%
    % INCLUDE THE CHAPTERS STARTING WITH THE NOMENCLATURE IF PRESENT
    %%%%%%%%%%%%%%%%%%%%%%%%%%%%%%%%%%%%%%%%%%%%%%%%%%%%%%%%%%%%%%%%%%%%%%%%%%%%%%%%%%%%%%%%%%%%%%%%%%%%%
    \include{front-matter}
    \chapter{Introduction} \label{ch:introduction}
\todo{Need a statement somewhere that more work has gone into systems than
workflows per se.}

Many civilizations tell an origin story for the diversity of human language. The
common thread in different versions of this story is that humanity
originally spoke a single language and united together to build a tower so tall
that it could reach heaven. Some versions say that the builders used blocks of
a common size, while others say that the builders used timbers of a common
length. The end is the same in most versions: When God learns of the
tower, he punishes the builders by confusing them, and then spreads them across
the world. The tower is left unfinished, heaven is left untouched, and the
builders are left speaking different languages.

Through an amount of research effort roughly equally to the work required
to build a sky-high tower, modern linguistics has demonstrated the low
likelihood of an original, single human language. However, the insights gained
about the nature of language, human anatomy, learning and neuroscience from this
effort were very valuable in their own right because of what they enabled or
revealed in other research efforts.

As complex as human language may be, computer science may represent the ultimate
test of our ability to study diverse ecosystems with new languages and tools
under constant, continuous development. Language and tool diversity is a
benefit to computing because each new language or tool is designed purposely to
solve a new problem, or to solve an old problem in a new way. This allows for
the entire technology stack to  be layered, optimized, and deployed in ways
specifically designed to exploit favorable conditions in complex systems. Case
in point, older programming languages such as Fortran and Kobol did not lose
popularity because of divine intervention. They lost popularity because of
economic forces that drove the development and adoption of more portable and
expressive system languages, such as C. Fortran and Kobol are still used in
places where they make sense, including high performance computing and finance,
but better tools are used where Fortran and Kobol are less than optimal.

One important class of programming languages and tools includes those that
can be combined with data to streamline and, in many cases, automate the
execution of tasks and processes. The major advantage of these tools is that
they make previously cumbersome activities repeatable and highly efficient.
This class solves \textit{workflow problems}, and is especially noteworthy
because of the poorly understood panoply of tools found in this space.

This work examines workflow problems and assocciated technology under two
assumptions that can be seen in parallel with broader
computing ecosystem. Specifically it considers that 1) there are no preferred
universal languages or tools, and 2) a lack of standardization
in software solutions is common because it is beneficial. Based
on these assumptions, this work shows that 
\begin{itemize}
  \item the workflow technology space is well covered by different types
  of systems,
  \item that an ontological treatment can be used to create a map of
  workflows and workflow management systems,
  \item and that this map can be used for next generation
  challenges such system interoperability and decision making.
\end{itemize}

The present chapter provides a thorough introduction of the workflow problem
space as well as some philosopical background to prepare the reader. Chapter
\ref{ch:ontologies} discusses ontologies, associated tools, and ontological
models relevant to workflows. Interoperability enabled solely through
ontological considerations is presented in chapter
\ref{ch:interoperability}. Chapter \ref{ch:eclipse-ice} introduces the
Eclipse Integrated Computational Environment (Eclipse ICE), which acted as the
primary model and served as a starting point for much of the
ontological and technical work. Chapter \ref{ch:blockchain} details a new model
data management and provenance capture for scientific workflows that embodies
the principles shared herein. A final summary and discussion of the
value of and opportunities for future work are presented in chapter
\ref{ch:conclusions}.

\subsubsection{Content sources}

The content in this document is largely based on separately published papers
that were collected, expanded, and edited for the purposes of better supporting
the argumentative stance of a thesis, and the formatting requirements of
the graduate school. The introductory text in this chapter is largely based on
work published previously in the Open Source Supercomputing workshop,
\cite{billings_toward_2017}. The content of chapter \ref{ch:eclipse-ice} was
adapted from a manuscript in the journal Software X,
\cite{billings_eclipse_2017}. The data management system in chapter
\ref{ch:blockchain} includes work presented as an invited talk at the First
International Workshop on Practical Reproducible Evaluation of Computer Systems
(P-RECS'18) with additional content on new work and the software system, the
\textit{Basic Artifact Tracking System (BATS)}, which has entered production
use. Additional content has been adapted from slides presented at international
conferences and workshops, as well as committee meetings.

The ontological and classification work presented in chapters
\ref{ch:ontologies} and \ref{ch:interoperability} is completely new, and at
time of this writing has not been published in manuscript form in
workshop, conference, or journal. However, the full source of the ontologies
and code has been made available on Github.com in the Eclipse ICE repository,
\cite{billings_ice}.

\todo{FIX the Eclipse ICE reference needs to be updated to
point to Software X}
\todo{the GitHub Eclipse ICE reference needs to be checked}

\section{Workflows}

\todo{Review CBB section}
\textit{The role that CBBs can play are at a system level. The section can be
reframed based on that.}

\baseInclude{pubs/workflows-paper/src/introduction}
\baseInclude{pubs/workflows-paper/src/workflows-review}
\baseInclude{pubs/workflows-paper/src/experience}
\baseInclude{pubs/workflows-paper/src/common}
\baseInclude{pubs/workflows-paper/src/buildingblocks}

\section{Summary}

The previous sections illustrate the complexity and diversity of workflow
technologies. Having amassed such data on the topic, it is tempting to develop
a new or adopt an existing definition of ``workflow'' and ``workflow system.''
However, settling on a single, simple definition has not worked well in the past
for a wide enough cross section of the community to meet future needs as workflows
begin to integrate experimental, computational, and analytical processes at
larger scales. Even more rigorous methods that attempt to create relevant
taxonomies are restricted to a single community, such as grid workflows in the
case of Yu and Buyya.

It is highly desirable to develop a deeper understanding of the similarities and
differences between workflows and related systems for several reasons. First, if
unnecessary duplication can be avoided and a greater understanding gained, they
should be to help with decision making and resource allocation. Second, a
deeper understanding may make it possible to do new, highly desirable things
with workflow management systems. Finally, it may reveal new ways to improve or
use related technologies including data management, machine learning, and
artificial intelligence.

Ontologies are efficient tools for gaining such an understanding as they
can formally catalog all of the different properties relevant to gaining
knowledge in a given topical area. This can be done in both human and machine
readable ways. The following chapters provide just such an analysis. However,
before turning to ontological considerations, and in an effort to better
understand the origins of the questions at the core of this thesis, it is
important to look at an interesting and somewhat unique workflow management
system: The Eclipse Integrated Computational Environment.


    \chapter{Ontologies} \label{ch:ontologies}

*What is an ontology?

*Ontology versus taxonomy

*Why do ontologies matter here?

*What does this look like in RDF?

*How can this be written in plain text, without invoking RDF?

\section{Case Study: A professor, a businessman, and a pilot}

\section{The Resource Description Framework}

*What is RDF?

\subsection{RDF Schema (RDFS)}

\subsection{Web Ontology Language (OWL)}

\subsection{Our case study in RDF}

\section{A Workflow Ontology}

A workflow may be defined as a collection of tasks that are executed in some order by human and non-human actors. A workflow problem can then be defined as any problem that is solved the the execution of a specific workflow. Many systems exist that can execute workflows encoded in one or more \textit{description formats} for both business and scientific problems. There are predominantly three types of scientific workflows: High-throughput \cite{}, Modeling and Simulation \cite{}, and Analysis \cite{}.

Workflow problems of any of type can be decomposed into three required components: The workflow description, the \textit{workflow engine} that executes the workflow based on the description, and the data required to fully describe and execute the workflow. The latter may include - but does not necessarily require - metadata that describes the contents of the data itself, bulk data including values and quantities of interest used in the workflow. (For the purposes of this work, it is sufficient to consider provenance information as a type of metadata.)
    \chapter{The Eclipse Integrated Computational Environment} \label{ch:eclipse-ice}

\baseInclude{pubs/ice-softwarex-2017/src/content}

\section{Compatibility and Other Issues with the Eclipse Integrated Computational Environment}

Practical systems exhibit many properties not found in theoretical systems. It is possible to see examples of these properties and differences by examining the development of the Eclipse Integrated Computational Environment (Eclipse ICE) version 3.0 compared to the existing version 2.0 framework. An existing system, such as Eclipse ICE 2.0 likely has a large number of features, requirements, bug fixes, and - if we are being completely honest - ``hacks'' that were necessary to acheive desired functionality. These could be very obvious, such as the fact that Eclipse ICE 2.0 does not use the Resource Description Framework (RDF) whereas version 3.0 does. Several examples of these types of problems are discussed below to illustrate the nature of compability issues between systems, including systems developed by the same authors.

\subsection{Changes in data structure design}

The aforementioned example of ICE 3.0 using RDF to describe its data structures is an obvious example of a difference between it and a legacy system such as ICE 2.0. At a design level, this means that ICE 3.0 has data structures that are described completely declaratively and ontologically in RDF, the RDF Schema languages (RDFS), or the Web Ontology Language (OWL). The definitions of the data structures are defined in OWL, which itself is defined in RDF, and \textit{instances} of these data structures are defined in RDF as well. The ontology exists in one OWL-RDF file, and the instances exist in their own files that import the base ontology. 

ICE 3.0 not only has different data structures, but it also uses uses a third party library, Apache Jena, to implement these data structures and provide useful services, such as mapping to HTTP/HTTPS servers or reading and writing to disks. ICE 2.0 used a custom implementation of its data structures, which were originally transcoded from UML to Java and updated subsequently by hand. 

Other differences are far more subtle and demanding in their detail. Data structures in ICE 2.0 manage their own notifications. Observers can \textit{register} as listeners to a data structure in ICE 2.0 and that data structure will notify the observer when it changes. This is a very low-level implementation of the implementation of the observer pattern that was designed to 1) facilitate live updates in user interfaces, and 2) to manage dependencies between data structures asynchronously. The latter case made it possible for a data structure to change its own value in response to a change made to a second data structure that was observing that was modified in the user interface. Both data structures would then asynchronously update their own observers and the user interface would finally update dynamically. Such a complicated case as this is commonly used in ICE 2.0, but live updates to the user interface happen with every workflow task.

RDF and OWL models in Apache Jena (and thus ICE 3.0) are only available at the highest level of the data model and they are very coarse grained: they only say that the model changed, not what data structure in the model changed. This is a completely valid way to handle update notifications, but in contrast to the model of ICE 2.0 it requires a full reload of the data structures by clients, including user interfaces. It has the distinct advantage of using far less resources than the ICE 2.0 model, namely that it does not launch separate threads for each update, but it comes at the possible cost of an expensive reload with every significant update. However, there are numerous strategies for mitigating the performance cost of a large reload.

The implications of a subtle change like this can be far reaching. For example, ICE 2.0 workflows that execute in ICE 3.0 will need to be broken into smaller pieces that divide any regions of dependency management into distinct steps. Alternatively, a Decorator pattern could be implemented around the model to capture function calls and thereby identify which elements of the data model updated. This would, in effect, acheive the properties of the ICE 2.0 model using the new model provided through Apache Jena.

    \chapter{Scientific Workflow Ontology}
\label{ch:workflow-ontology}

The tools, techniques, and background knowledge provided in the previous
chapters makes it possible to answer important questions about an eclectic mix
of workflow technologies. These questions include ``What is a workflow?'' and
``Are workflow management systems conceptually the same?,'' ---all with the goal
of establishing whether the tools are varied or merely variegated. The first of
these two questions ---``What is a workflow?''--- is of particular interest in
this work because a better understanding of the different types of workflows will
present the opportunity to examine problems that were previously too costly in
manual labor or altogether impossible.

The following sections present a workflow ontology, \S
\ref{workflow-ont-section}, and the method used to develop it, \S
\ref{workflow-ont-method}. Finally, concrete examples of the application of this
ontology to workflows ```in the wild'' are provided to show its range as a
decision-making and scientific computing tool.

\section{Methodology}
\label{workflow-ont-method}

\todo{Talk about design philosophy?}

The scientific workflow ontology presented in \S \ref{workflow-ont-section} was
developed by considering workflows in two environments: i) the context
of problems they solve and ii) as entities that are executed by workflow
management systems.

\subsection{Workflow Problems}

Workflows are most interesting in the context of problems they solve. As
Chapter \ref{ch:introduction} demonstrates, because of the large number of these
problems, it can be very difficult to write an all-encompassing
definition of a scientific workflow by looking only at the workflows directly. 

One classic way to solve calculus problems without an obvious solution is the
method of change of variables in which new variables related to the original
variables by some relationship are used in place of the originals. Changing
variables makes it possible to mask certain types of complexity to reveal direct
methods of solving the problem. An analogy to this method can be used to study
scientific workflows. Specifically, seeking a definition for workflow problems
instead of workflows can make it possible to find a definition of scientific
workflows by ``solving'' for it. 

It is sufficient for the purpose of this work to define a workflow problem
by building an ontological model based on the description of workflow
management systems, workflows, and data in other chapters. Workflow problems of
any of type can be decomposed into three required components: i) The workflow
description, ii) the workflow engine or management system that executes the
workflow based on the description, and iii) the data required to fully describe
and execute the workflow. The latter may include ---but does not necessarily
require--- metadata that describes the contents of the data itself, bulk data
including values and quantities of interest used in the workflow. (For the purposes of
this work, it is sufficient to consider provenance information as a type of
metadata.) A workflow problem, then, is one that is solved by providing a
workflow description to a workflow management system with all pertinent data in
hand. Thus, by examining the set of workflow problems and workflow management
systems, while allowing data to act as a kind of free variable, it is possible
to describe the set of workflows completely.

This is an empirical way of thinking about workflows that results in an emergent
definition, versus a prescribed one. This method accepts the community will
move as it sees fit, but asserts (quite strongly) that progress can still be
made by considering what exists collectively. The method is additive since
any new workflow management system can be studied to learn about the
workflows it supports, and the description of those workflows can be added to
the model created by the original effort. As the model grows, it will enclose a
larger area of the workflow space, resulting in the emergence of a new or
updated description of the set of abstract workflows.

This method responds well to a modeling treatment, and, indeed, may be described
as a modeling method. All the languages and tools in Chapter \ref{ch:ontologies}
can be applied.

\subsection{Referenced Workflow Management Systems}

Workflows from several workflow management systems were examined as part of this
work. This include workflows from Eclipse ICE, Taverna
\cite{wolstencroft_taverna_2013}, Triquetrum \cite{brooks_introducing_2016},
Pegasus \cite{noauthor_pegasus_nodate}, the Common Workflow Language
\cite{noauthor_common-workflow-language:_2018}, Cylc
\cite{noauthor_cylc_nodate}, Chiron \cite{ ogasawara_chiron:_nodate}, Moteur
\cite{glatard_flexible_2008}, and SAW \cite{clay_incorporating_2015}. One
unnamed heirarchical workflow management system from Argonne National
Laboratory was also reviewed.

\section{Workflow Ontology}
\label{workflow-ont-section}

This section describes an ontology for scientific workflows created using the
method and philosophy described in the previous section. Classes, object
properties, and data properties are listed subsequently. The full OWL ontology,
(created in Prot\'eg\'e), is provided as a TURTL file in Appendix
\ref{app:full-ontology} to preserve space for the narrative here. The full
TURTL file includes some axioms not discussed here. 

Unlike the example in Chapter \ref{ch:ontologies}, no individuals are described
in this section. The following figures are Graphviz visualizations
of subgraphs of the main ontology graph pulled from Prot\'eg\'e using its OntoGraf
plugin. These figures illustrate the relationships among the core classes and
properties of the ontology. Table \ref{ont-stats-table} summarizes numerous
metrics of the ontology.

\begin{table}[H]
\begin{tabularx}{\textwidth}{|X|X|}
\hline
Triple count & 241 \tabularnewline\hline
Axiom count	& 161	\tabularnewline\hline
Logical axiom count	& 53	\tabularnewline\hline
Declaration axioms count &	40	\tabularnewline\hline
Class count	& 25	\tabularnewline\hline
Object property count	& 9	\tabularnewline\hline
Data property count	& 3	\tabularnewline\hline
Individual count &	0	\tabularnewline\hline
Annotation property count	& 8 \tabularnewline\hline
SubClassOf	& 21		\tabularnewline\hline
DisjointClasses  &	5 \tabularnewline\hline
SubObjectPropertyOf	& 3 \tabularnewline\hline
ObjectPropertyDomain &	10	\tabularnewline\hline
ObjectPropertyRange	& 7 \tabularnewline\hline
SubDataPropertyOf	& 1 \tabularnewline\hline
DataPropertyDomain	& 4	\tabularnewline\hline
DataPropertyRange &	2 \tabularnewline\hline
AnnotationAssertion	& 68 \tabularnewline\hline
\end{tabularx}
\caption{Ontology Statistics}
\label{ont-stats-table}
\end{table}

The full ontology is also preserved in the Eclipse ICE GitHub repository
\cite{billings_ice_2019}.

%%% Include generated ontology text.
\baseInclude{chapters/ice-workflows.out.tex}

\section{Examples and Applications}
\label{workflow-ont-examples}

This section shows examples and applications of workflows marked up as
instances of the workflow ontology. The full RDF listings for all
examples are provided in the appendices.

\subsection{Basic File Move}
\label{move-workflow}

The first example is a very simple workflow showing the move of a file from
one location to another. The workflow description executes a single task, which
is specialized to move a file using a dedicated action type. The action type
uses a Java class as its target, and it uses simple string values for file name
input and output.

This example is important because it shows how simple it is to wire together a
straightforward workflow. The relationships among workflow descriptions,
tasks, and actions create a directed acyclic graph for this type of workflow.
However, it is also important because file transfer is a common task in many
workflows, often occurring as a subworkflow that executes before and after
other tasks or before and after the main workflow.

Figure \ref{move-workflows} shows a version of this example that has been
slightly edited to better fit the page. The full TURTL version of this workflow
is available in the appendices, \S \ref{app:move-workflow}.

\begin{figure}[htbp]
\centering
\baseIncludegraphics{figures/moveWorkflow.png}
\caption{The basic file move workflow.}
\label{move-workflows}
\end{figure}

\subsection{Combining Cycles and Loops}

The next example covers a more complicated use case that is arguably uncommon
in more popular workflow engines: cycles and loops. This example, depicted in
figure \ref{cycle-loop-test}, describes a simple workflow were 50 files are
created using a threshold limiter and then deleted through a standard
50 iteration loop. A threshold limiter limits the amount of something --- in
this case the number of files created --- based on a threshold.

This workflow is a simple model, but it is a good analog to more complicated
systems in which data is gathered until a threshold is crossed, such as
the number of counts from a detector, and then the set of files are processed in a loop. A
threshold is well modeled by a cycle because the system waits until the
threshold is met, which requires a periodic (cyclic) check against the limit.

The examples contains two tasks, \#loopTask and \#cycleTask, with the \#loopTask
being dependent on \#cycleTask. The dependency is because \#cycleTask
creates the files, which must all exist before the \#loopTask is executed. The
action of the cycle task is to create a file using the Linux ``touch'' command,
and its condition is to created files until the number of files in the directory
is greater than 50. It checks the number of files using a Java program called
``fileCounter.'' When this task is complete, the dependency for \#loopTask is
satisfied and it can execute its action --- the Linux ``rm'' command to remove a
file. The condition on the loop is that its action is executed for fifty
iterations specified by the lowerBound, upperBound, and stepSize properties.

Workflows such as this are easy to model in workflow systems where loops and
cycles are supported. It is also possible to execute this workflow in systems
that do not directly support those constructs by decomposing it into smaller
workflows since the number of files is fixed. The \#cycleTask can be
executed as a linear graph of fifty separate system checks. The \#loopTask can
be unrolled to fifty separate executions of the remove command. However, limited
polling and unrolling, respectively, only work in cases where the total number
of iterations is fixed.

\begin{figure}[htbp]
\centering
\baseIncludegraphics{figures/cycle-loop-test.png}
\caption{Combined cycle and loop workflow example.}
\label{cycle-loop-test}
\end{figure}

The full TURTL version of this workflow is available in the appendices, \S
\ref{app:cycle-loop-test}.

\subsection{Pegasus Split Example}

The Pegasus workflow management system is especially good at executing large,
parallel workflows. The Pegasus website and documentation provide a simple
example of splitting files into parts in parallel
\cite{noauthor_workflow_nodate}. The workflow model of Pegasus is very well
paired with the workflow ontology, and mapping this example from its source can
be accomplished in a very straightforward fashion.

Figure \ref{pegasus-split-workflow} is graph of the Pegasus splitting example
showing only the most essential parts, namely, tasks, actions, action types, and
properties from the workflow ontology. Files and parameters in Pegasus map
directly to properties in the ontology, with the jobs that are executed against
them mapping to tasks and actions. This example illustrates how quickly even a
simple workflow can become too hard to easily visualize in its entire scope, so
figure \ref{pegasus-comparison} shows i) a greatly slimmed down version of the
worfklow graph that contains only the tasks compared against ii) the original
graph of the image from the Pegasus website.

One important difference of the workflow ontology that is highlighted by
Figure \ref{pegasus-comparison} is that it models workflow tasks and their
dependencies, whereas other workflow models are focused on data flow. This is
highlighted by the inversion of the arrow heads across the sides of the image.
In part a), the arrows point from the tasks at the top of the dependency chain
down to the initial task that has no dependencies against it. The opposite is
true in part b), which shows the initial task with no dependencies generating
output that is fed into the tasks where it is required. 

Both views are correct: They are equally valid perspectives are akin
to saying that ``The donkey pulls the cart'' and ``The cart is pulled by the
donkey.'' As long as the interpreter gets the point that the cart needs to move,
there is not problem. This inverted graph phenomenon is also witnessed in
provenance graphs when compared against workflows they describe (see \S
\ref{provenance-background}).

\begin{figure}[htbp]
\centering
\baseIncludegraphics{figures/pegasusSplit.png}
\caption{Only the most essential elements of the Pegasus split
example as marked up in the workflow ontology. This level of detail ---far
removed from the full details--- shows the high number of facts captured with
semantic models.}
\label{pegasus-split-workflow}
\end{figure}

\begin{figure}[htbp]
\centering
\baseIncludegraphics{figures/pegasus-comparison.png}
\caption{Graph of Figure \ref{pegasus-split-workflow} on the left with only
the tasks shown compared with the original Pegasus split example from the
website on the right.}
\label{pegasus-comparison}
\end{figure}

The full TURTL version of this workflow is available in the appendices, \S
\ref{app:pegasus-workflow}.

\subsection{Eclipse ICE II/III Task Model}

Eclipse ICE, covered extensively in Chapter \ref{ch:eclipse-ice},
uses a somewhat unique view of workflows in its workflow model. Workflows
in Eclipse ICE are executed as finite state machines that can include human
feedback, conditional branching, and error conditions. This makes it possible
to describe workflows in a conceptually abstract way using state machine
theory. It remains possible to model the execution flow of these state machines
as directed acyclic graphs, which can be observed in Figure \ref{ice-workflow}.
When cast into the workflow ontology, it is clear that the abstract workflow
executed by ICE is directed and acyclic, even if its instances may execute with
cycles and other conditions.

ICE workflows start by setting up the Form used to collect
\#workflowProperties, which is the \#setupFormTask. The workflow continues
through multiple tasks that depend on the initial Form, as well as other tasks,
and which cause state changes within the system. On state changes, the task
executes the \#stateChangeAction to update the system state, reconfigure data,
and prepare the next task. User feedback is required, and the \#submitForm task
will wait until this condition is met before transitioning to review and
processing. Like the earlier examples, arrows point from tasks to previously
executed tasks through the ``dependsOn'' property such that the last task,
\#processTask, shows up at the top and does not appear to feed any other tasks.

The system goes from one task to another through the state changes. Each state
change configures the system so it can execute its next task and ensures
that all previous steps were executed properly. Thus, the internal logic of when
a task is complete remains internal to the workflow, and no external logic is
required to satisfy task completion.

This example shows a very interesting relationship between the \#hasCondition
and \#dependsOn object properties in the ontology: Conditions are merely
special, internal dependencies that are managed directly by the task instead of the
workflow engine. Conditions execute small ``micro-workflows'' within the task
that affect its completion, while the workflow engine manages its completion by
first executing the tasks on which it depends. Thus, conditions can be thought
of as merely subgraphs of the larger workflow that sit between a task and its
dependencies.

\begin{figure}[htbp]
\centering
\baseIncludegraphics{figures/iceWorkflow.png}
\caption{The tasks, actions, and states of the standard workflow model of
Eclipse ICE, as graphed using the workflow ontology.}
\label{ice-workflow}
\end{figure}

The full TURTL version of this workflow is available in the appendices, \S
\ref{app:ice-workflow}.

\subsection{Neutron Scattering User Workflow and Data Pipeline}

The Spallation Neutron Source and the High-Flux Isotope Reactor operated by Oak
Ridge National Laboratory are the premiere neutron sources in the United States.
Figure \ref{neutron-workflow} shows the idealized, conceptual workflow that
users can expect while performing experiments at the
facility.\footnote{Image and information courtesy of the author.} This is a good
example of the conceptual workflows mentioned in Chapter \ref{ch:introduction},
\S \ref{workflows}, because it cannot be executed and represents high-level,
idealized tasks.

\begin{figure}[htbp]
\centering
\baseIncludegraphics{figures/neutronWorkflow.png}
\caption{Conceptual neutron scattering user workflow and data pipeline.}
\label{neutron-workflow}
\end{figure}

Figure \ref{neutron-workflow-graph} shows the graph of this workflow when
translated to the workflow ontology. One immediate observation is that
conceptual workflows may not require a deep translation to be understandable
and that in some cases only tasks are needed to describe conceptual workflows.
Still, it is useful to look at conceptual worfklows in the same framework as
concrete workflows to understand how conceptual workflows can evolve to be
concrete and executable in the future.

Another important observation is that this workflow is cyclic. It is tempting to
believe that the Design of Experiments task is the first task because it is the
top-most, left-most task in the diagram, which to English speakers may suggest
that it is special. It is true that in some cases Design of Experiments may be
the first task, but many users walking through this workflow start at other
points, such as ``HPC, Modeling/Simulation, AI/ML'' because they approach
the problem from a theoretical perspective. There are several others cycles
between the various parts of the graph, such as the ``Data Reduction'' to
``Data Curation and Archival'' to ``HPC, Modeling/Simulation, AI/ML'' cycle.

\begin{figure}[htbp]
\centering
\baseIncludegraphics{figures/neutronWorkflow-graph.png}
\caption{Conceptual neutron scattering user workflow and data pipeline
graphed in the workflow ontology.}
\label{neutron-workflow-graph}
\end{figure}

The full TURTL version of this workflow is available in the appendices, \S
\ref{app:neutron-workflow}.

\section{Summary}
\label{workflows-ont-summary}

This chapter presents the methodology and reasoning behind creating a workflow
ontology using the methods discussed in Chapter \ref{ch:ontologies}, the
ontology for scientific workflows itself, and five examples of scientific
workflows mapped or translated into instance graphs of the ontology. The
examples, in particular, demonstrate the central thesis of this work that a
comprehensive metamodel of scientific workflows could describe multiple types of
scientific workflows. This includes high-throughput (Pegasus); modeling and
simulation (Eclipse ICE); iterative workflows that run repeated, looping, and
cyclic tasks (the cyclic looping example); and purely conceptual workflows that
are not executed by machines; but remain important for communication and other
work (neutron scattering user workflow and data pipeline). The examples
further demonstrate that common patterns can be described by the ontology
in a common way that is agnostic to the underlying platform, much like
design patterns in programming (Move Workflow example). 

What remains is the final question of this thesis: If scientific workflows can
be uniformly described, what are the implications for interoperability? Chapter
\ref{ch:interoperability} will seek to address this question by examining the
many different facets of interoperability.

    \chapter{Ontological Interoperability} \label{ch:interoperability}

Discuss wrapping workflow engines instead of native execution
    \chapter{Workflows on the Blockchain} \label{ch:blockchain}

Important realization with data in an open world is how do you manage it? Actors
must be free to\ldots
*Describe what they have
*Store bulk data
*Share what they have

This chapter describes a system that was developed as part of this thesis work
to address these concerns for ICE III.

\baseInclude{pubs/billings-workflows-blockchain/content}
    \chapter{Conclusions}\label{ch:conclusions}

Epistomological
POINT 1 from about next-generational problems, etc.

NOTE change in strategy from CBB, also Kingmakers and Building Microservices.

\baseInclude{pubs/workflows-paper/src/discussion}

\listoftodos
    %%%%%%%%%%%%%%%%%%%%%%%%%%%%%%%%%%%%%%%%%%%%%%%%%%%%%%%%%%%%%%%%%%%%%%%%%%%%%%%%%%%%%%%%%%%%%%%%%%%%%
    % BIBLIOGRAPHY
    %%%%%%%%%%%%%%%%%%%%%%%%%%%%%%%%%%%%%%%%%%%%%%%%%%%%%%%%%%%%%%%%%%%%%%%%%%%%%%%%%%%%%%%%%%%%%%%%%%%%%
    \makeBibliographyPage % make the bibliography title page
\newpage

% To make the bibliography, use \utbiblio{#1}{}{} command. Always use "#1" for the first entry. The second entry is your bibliography style, and the third entry is the name of your bibliography file (.bib file extension) 
% bibliography style - recommend using apalike-doi as it hyperlinks DOIs
% Be sure to run BibTeX in order to generate the bibliography correctly.

\utbiblio{#1}{apalike}{bib}

    %%%%%%%%%%%%%%%%%%%%%%%%%%%%%%%%%%%%%%%%%%%%%%%%%%%%%%%%%%%%%%%%%%%%%%%%%%%%%%%%%%%%%%%%%%%%%%%%%%%%%
    % APPENDIX - OPTIONAL - COMMENT OUT IF NOT NEEDED
    %%%%%%%%%%%%%%%%%%%%%%%%%%%%%%%%%%%%%%%%%%%%%%%%%%%%%%%%%%%%%%%%%%%%%%%%%%%%%%%%%%%%%%%%%%%%%%%%%%%%%
    
    \makeAppendixPage{2}   % Input the number of appendices
    \appendix    
   % 
\section{Summary of Equations}
some text here
\subsection{Cartesian}
some equations here

\subsection{Cylindrical}
some equations also here
%    
\section{Summary of Stuff}
some text here
\subsection{More Things}
some equations here

\subsection{Other Aspects}
some equations also here
    %%%%%%%%%%%%%%%%%%%%%%%%%%%%%%%%%%%%%%%%%%%%%%%%%%%%%%%%%%%%%%%%%%%%%%%%%%%%%%%%%%%%%%%%%%%%%%%%%%%%%
    % A VITA IS REQUIRED
    %%%%%%%%%%%%%%%%%%%%%%%%%%%%%%%%%%%%%%%%%%%%%%%%%%%%%%%%%%%%%%%%%%%%%%%%%%%%%%%%%%%%%%%%%%%%%%%%%%%%%
    \addToTOC{Vita}
    \chapter*{Vita} \label{ch:vita}

Jay Jay Billings is a Research Scientist in the Neutron Scattering
Division (NSD) and Computer Science and Mathematics Division (CSMD) at Oak Ridge
National Laboratory. He leads the Scientific Computing and Software Engineering group
in NSD and Research Software Engineering group in CSMD. He holds the Bachelor’s
Degree in Physics from Virginia Tech, class of 2005, and the Master of Science
in Theoretical Astrophysics from the University of Tennessee, class of 2008.
Mr. Billings’ research focuses on the design and implementation of modeling and
simulation tools for energy science, a large part of which has been related to
the study of scientific workflows in an HPC context.

At Oak Ridge National Laboratory, Mr. Billings is leading the Scientific
Software Initiative within CSMD and is a member of the Software Council, both of
which are taking a new look at the way ORNL develops software for the Department of
Energy. He is a founding member and current chair of the Science Working Group
at the Eclipse Foundation, where he also leads the Eclipse Integrated
Computational Environment and the Eclipse Advanced Visualization Project. Mr.
Billings was also appointed to the Eclipse Architecture Council in 2016, and is
a mentor for several additional Eclipse projects. He is a member of the
Association for Computing Machinery.

Mr. Billings has been funded by the Department of Energy Offices of Nuclear
Energy, Energy Efficiency and Renewable Energy, Advanced Scientific Computing
Research, Basic Energy Sciences, and Advanced Manufacturing.

In addition to his day job, Mr. Billings is a candidate for the PhD in Energy
Science from the Bredesen Center for Interdisciplinary Research and Education at
the University of Tennessee. He spends his spare time with his family, and
singing.

\end{document}

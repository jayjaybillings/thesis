\chapter*{Abstract}\label{ch:abstract}

Scientific workflows exist in many different domains and for many different
computing platforms. As these systems have proliferated, they have also become
increasingly complex and harder to maintain. Furthermore, these systems often
exist as self-sufficient islands of capability that can be over-specialized and
locked into a specific domain. Some commonality exists and three major workflow
types are readily apparent in (i) modeling and simulation, (ii) high-throughput
data analysis, and (iii) optimization. A far more detailed understanding of
different workflow types is required to determine how large, interdisciplinary
workflows that span the types and multiple computing facilities can be created
and executed. This work presents a new model of scientific workflows that
attempts to create such an understanding with a formal, machine-readable
ontology that can be used to answer design questions about interoperability for
workflows that need to be executed across distributed workflow management
systems. Example instances are presented for simple workflows that do not
require decision making, more complicated workflows that can split decision
making between external agents and internal state transitions in finite state
machines, and purely conceptual workflows that represent notional if not exactly
executable workflows purely for communicating ideas. Finally, a perspective on
interoperability for workflow systems is presented in the context of the
ontology.
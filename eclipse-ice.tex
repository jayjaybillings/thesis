\emph{This article is to be submitted to the Journal ``Software X'' and
is thus presented in the journal's required format. Mostly.}

\section{Notice of Copyright}\label{notice-of-copyright}

This manuscript has been authored by UT-Battelle, LLC under Contract No.
DEAC05-00OR22725 with the U.S. Department of Energy. The United States
Government retains and the publisher, by accepting the article for
publication, acknowledges that the United States Government retains a
nonexclusive, paid-up, irrevocable, world-wide license to publish or
reproduce the published form of this manuscript, or allow others to do
so, for United States Government purposes. The Department of Energy will
provide public access to these results of federally sponsored research
in accordance with the DOE Public Access Plan
(http://energy.gov/downloads/doe-public-access-plan ).

\section{Abstract}\label{abstract}

FIXME!

FIXME! - keyword 1 ,keyword 2 ,keyword 3

\section{Motivation and Significance}\label{motivation-and-significance}

High performance modeling and simulation software is hard to use. Much
of the challenge comes from the inherent nature of the science under
consideration, but perhaps just as much of the problem comes from
focusing very heavily on raw compute performance. Focusing on
performance is understandable, given the field, but it leaves the rest
of us to figure out how to actually use the software in a way that
scales for a much larger set of users, many of whom may be novices or
migrate between software packages regularly.

** REMOVE or merge with text below!**

In previous work, Billings et. al., discovered through requirements
gathering interviews that many of the difficulties using high
performance modeling and simulation software fall broadly into five
distinct categories, \cite{billings_cbhpc}. This includes input
generation and preprocessing (also called ``model setup''); job
execution and monitoring; postprocessing, visualization and data
analysis; data management; and customizing the software. There are many
tools that address these problems individually, but the same work found
that the excess number and specialization of these tools also contribute
to the learning curve.

Efforts to address these five issues typically fall in with general
purpose scientific workflow tools like Kepler, \cite{kepler}, or are
reduced to myopic tools that satisfy some set of requirements for a
single piece of software or platform. That is, the proposed solution to
this problem is often to shoe-horn it into existing workflow tools that
are so general that they focus on nothing in particular or to ignore the
general problem entirely and deploy a completely tailored solution for
the given application. These are opposing extremes, but a
middle-of-the-road solution is also possible. A workflow engine could be
developed that limits its scope to High-Performance Computing (HPC) and
to the set of possible workflows that come from the previously mentioned
five activities. A rich enough Application Programming Interface (API)
could be exposed so that highly customized solutions could still be made
based on this limited workflow engine with only a relatively minor
amount of additional development required.

It is not clear that one of these solutions is better than the others.
Practical requirements will ultimately dictate which way projects go.
This work considers a middle ground solution and presents the Eclipse
Integrated Computational Environment (ICE) as proof that it is possible
to create such a system. Specifically, we show * an architecture for
such a workflow system that satisfies the model of workflows presented
below in an extensible way. * that such a system can be cross-platform
for workstations and simultaneously deployable on the web. * that smart
design decisions enable not only authoring code for simulation software,
but also make it possible for the system to extend itself, thereby
enabling heavy customization.

\textbf{FIXME! Needs to transition smoothly!}

\subsubsection{Workflow Model}\label{workflow-model}

Computational scientists perform a variety of tasks with respect to
modeling and simulation that can be abstracted into a lightweight
theoretical framework broadly based on five high-level
\emph{activities}: creating input, executing jobs, analyzing results,
managing data, and modifying code. The same computational scientist
would most likely find these activities difficult for all codes with
which they lack experience, whereas with their own codes or those with
which they are most familiar these tasks may be so easy that they are
taken for granted. Any particular combination of these activities across
one or more scientific software packages or codes results in a unique
\emph{workflow}. Such a workflow is normally, but not always, requested
by a human user and orchestrated by a \emph{Workflow Management System
(WMS)}.

The most obvious workflow for any individual simulation code or
collection of codes is to string the activities together: the user's
workflow is to create the input, launch the job, perform some analysis
and manage their data, possibly modifying their code in the process.
There are many other combinations though, such as re-running jobs with
conditions or modifications or analyzing someone else's data. (See
footnote 1.)

\textbf{Creating input} is the process of describing the physical model
or state of a system that will be simulated. This could include creating
an input file or file(s), but could also include external calls to a
process to configure a running program. In most situations a
computational scientist will modify existing input or create new input
from a template. ``Input'' generally includes runtime parameters for the
simulation framework (tolerances, etc.); configuration options (data
locations, output locations, module configurations, etc.); material
properties for the materials that will be simulated; and a
discretization of the simulation space (mesh, grid, particle
distribution, etc.). The collection of all required input can be quite
large and may go by many names such as ``input set,'' ``input package,''
``problem,'' or, simply, ``input.'' Often the set of input files will be
described in a ``main'' input file that acts as a kind of manifest,
describing and providing links to all necessary information for the
given problem.

In this work, it should be assumed unless otherwise noted that ``input''
refers to the entire set of input, not a single file.

\textbf{Executing jobs} or ``running the workflow'' in this context is
the process of performing some calculation based on the input with a
simulation code or framework. Modeling and simulation codes are normally
run locally for small jobs or development. Large simulations typically
require equally large hardware resources that are utilized remotely
through Secure Shell (SSH) connections or similar protocols. Remote
execution requires moving the input in advance of the execution and
copying or moving the output to the user's machine, if possible and
desirable. (In many cases the output cannot be moved because it is too
large.)

Local and remote jobs are often monitored in one or more ways to
ascertain the status of job. This can include simply checking whether or
not execution has finished or monitoring the output of individual
quantities to examine the calculation state. The latter is often used to
detect calculation errors that will result in incorrect results. If such
results are found, the job is typically cancelled (``killed'') to save
compute resources and re-run later.

In this work, it should be assumed unless otherwise noted that
``executing a job'' includes monitoring that job in one or more ways,
possibly including real-time updates to visualizations.

\textbf{Analyzing results} includes executing special jobs to transform
data in one or more prescribed ways and producing artifacts with
scientific significance from the transformed data. This may include, for
example, post-processing results and visualizing the new data with
dedicated visualization tools. For many types of scientific computing
this includes viewing the results of a simulation on a mesh or grid and
extracting publication quality images or movies from that data. Other
cases may include analyzing results in preparation for follow
simulations or performing feature extraction, classification or
activities for machine learning and data mining.

While this has many similarities to executing a job, it is distinctly
different because the activity changes focus to satisfy the needs of a
human operator. Simple data reduction where the exact reduction is known
certainly qualifies as executing a job, but analysis of modeling and
simulation results is far from simple data reduction. It is generally a
far more interactive process for computational scientists.

\textbf{Managing data} includes moving, copying, storing, sharing or
otherwise interacting with data for or from simulations. This activity
is the most pervasive of all the activities because each of the others
requires interacting with data in some way. In many cases though, data
is still managed its own purposes, without performing a simulation,
generating new input, or analysing results. Examples include archiving
data, packaging data for publications, and updating values manually
(often in light of new information from publications).

\textbf{Modifying code} is not typically considered a part of a
scientific computing workflow. However, modeling and simulation use
cases often require users to explicitly modify code before execution,
such as with the computational fluid dynamics code Nek5000,
{[}Nek5000{]}; or to issue special re-build instructions. Many
computational scientists consider ``their workflow'' to be re-running
software after modifications for purely exploratory purposes. This may
be required if the model the author is manipulating can not be
configured as part of the input directly but can be easily manipulated
by hand.

\subsubsection{Comparison to Other
Models}\label{comparison-to-other-models}

This model of workflows differs significantly from those of similar
efforts in workflow science because it defines workflows in terms of
activities. Others in the literature define a workflow as a collection
of computing processes. For example, Yu and Buyya define grid workflows
as ``a collection of tasks that are processed on distributed resources
in a wel-defined order to accomplish a specific goal,'' {[}Yu, 2005{]}.
Others such as Pizzi et al. subscribe to similar definitions, {[}Pizzi,
2015{]}. This ``process'' view is acceptable where the workflow is
static and does not require additional human input or ``human in the
loop'' behavior after the all the initial human input is provided. That
is, a ``process-oriented'' definition is acceptable where all human
input is provided in advance. However, workflows within ICE are fully
interactive with regular call-backs to humans. It is simpler to discuss
``activities'' than it is to create a distinction between ``human
processes'' and ``computer processes.'' Focusing on ``activities'' over
processes (human or computer) also has the benefit of removing concrete
elements such as hardware properties or software details that distract
from details of workflows and WMSes per se. That is, considerations such
as memory usage and raw performance are important, but questions about
the abstract workflow or what the WMS should do are far more important.

\section{Software Description}\label{software-description}

ICE was specifically created to address the workflow model described
above, which is to say that it was created specifically to address
hands-on workflows for computational scientists. Users download and
execute ICE locally and it orchestrates workflows locally or remotely as
required. It provides a comprehensive workbench for modeling and
simulation that includes tools for workflows, visualization, data
management, and software development.

\subsection{Software Architecture}\label{software-architecture}

Workflows and tasks in ICE are not explicitly treated as trees -
Directed Acyclic Graphs, (DAGS) - as is common with grid workflow tools,
{[}Yu, 2005{]}. Instead, ICE's design is inspired by Representational
State Transfer (REST), {[}Fielding, 2000{]}.

Users initiate requests to create, edit, update or delete workflows from
the \emph{ICE Client} (the workbench). The list of available workflow
that can be created is provided dynamically to the \emph{ICE Client} by
the \emph{ICE Core}, which acts as a kind of server. This information is
provided dynamically since it often changes as run time based on the
configuration of available workflow components in the registry and on
persisted workflows that users have saved in their workspace.
Information about workflows are provided from the Core to the Client
through \emph{stateless Forms} that describe the workflow and provide
all the necessary information to understand \emph{what} should process
the workflow, (but not \emph{how} it should be processed). Once users
modify the description of the workflow in the Form to provide their
specific details, the Client dispatches a request to the Core to modify
or process the workflow. The Core then uses the information in the Form
to perform a service lookup for \emph{what} should process the workflow.

Workflows in ICE are encoded in and processed by \emph{Items} and each
workflow type is a subclass of Item. Items are registered dynamically
through a service registry in ICE, (see section \ref{framework}), and
provide the Forms for the ICE Core and Client. Since ICE's design is
highly object oriented, it is easiest to think of the Item class as a
description of an abstract workflow and an Item object (an instance of
the class) as a concrete workflow with all required execution details
provided by its Form.

Individual components of workflows, i.e. - workflow ``tasks'' or
``nodes'' - are either encoded directly in the workflow's subclass of
Item or provided as \emph{Actions} that are dynamically registered with
an \emph{Action Factory} and obtained at runtime. Table 1. describes the
differences between Items, Actions, and Forms. (See footnote 2.)

Table 1: Class Descriptions for Items, Forms and Actions

\begin{longtable}[c]{@{}lll@{}}
\toprule
Class & Class Description & Object Description\tabularnewline
\midrule
\endhead
Item & Java class with code to execute an abstract workflow. Provides a
Form. & Concrete workflow executor.\tabularnewline
Form & Description and template of the data needed for the Item to
process the workflow. & User-modified workflow data.\tabularnewline
Action & Java class for executing a specific task in the workflow. Used
by Items. & Concrete workflow task executor.\tabularnewline
\bottomrule
\end{longtable}

\subsubsection{WORKSPACES AND
PERSISTENCE}\label{workspaces-and-persistence}

ICE persists permanent copies of Forms to disk when workflows are
created and modified in a special directory called a \emph{workspace}.
Workspaces can contain projects, folders and files, including data,
code, input and output. ICE automatically manages local and remote (or
even local \emph{to} remote) transfers of data files when executing
workflows if the files are detected in the same directory of the
workspace as the workflow itself. For example, when executing a remote
job, ICE will automatically move the input file if it is specified in
the workflow and available in the project directory of the workspace.
All paths in ICEIf a workflow uses a relative link to a file in the
workspace, ICE will au Workflow persistence is managed in ICE by a
\emph{persistence provider} fordefault persistence provider uses JAX-RS,
{[}Burke, 2009{]}, to write

\subsubsection{ITEM-STATES}\label{item-states}

\subsubsection{SYSTEM-CALLS}\label{system-calls}

\subsubsection{SERVER MODE}\label{server-mode}

\subsubsection{JYTHON SHELL FOR LARGE
JOBS}\label{jython-shell-for-large-jobs}

\subsubsection{ENCODED WORKFLOWS}\label{encoded-workflows}

\subsubsection{VISUALIZATION}\label{visualization}

\subsubsection{DEVELOPMENT AND
SELF-HOSTING}\label{development-and-self-hosting}

TESTS!

\subsubsection{Framework}\label{framework}

ICE is an Eclipse Rich Client Platform (RCP) application and is
therefore based on Equixno, the reference implementation of the Open
Service Gateway Initiative (OSGi) framework specification, {[}McAffer,
2010{]}.

Resources bundle for workspace

PULLING FROM ECLIPSE (NOT a subsection)

\subsubsection{Comparison to Grid Workflow Management
Systems}\label{comparison-to-grid-workflow-management-systems}

\paragraph{ABSTRACT v CONCRETE}\label{abstract-v-concrete}

OO. Workspace files

\subsection{Software Functionalities}\label{software-functionalities}

ICE is used in the following scenarios\ldots{}

\begin{itemize}
\itemsep1pt\parskip0pt\parsep0pt
\item
  Rapid deployment of workbenches
\item
  Rapid deployment of domain-specific development tools
\item
  \ldots{}
\end{itemize}

\subsection{Sample Code}\label{sample-code}

Workflow Item examples:
https://github.com/eclipse/ice/tree/master/org.eclipse.ice.demo

Scripting examples: https://github.com/eclipse/ice/tree/master/examples

\section{Illustrative Examples}\label{illustrative-examples}

VIBE BJM REFLECTIVITY MOOSE

Figure 1 shows an example of the ICE workbench for a simple structural
mechanics problem solved with the MOOSE framework, {[}Gaston, 2009{]}.

\begin{figure}[htbp]
\centering
\includegraphics{\{\{ site.url \}\}/\{\{site.baseurl\}\}/images/ice-moose.png}
\caption{ice-moose}
\end{figure}

JADE

Example of how far it can be customized - Reactor Analyzer (

\section{Impact}\label{impact}

UI work limiting adoption

Moving to the web5i

\section{Future Work}\label{future-work}

JAX-RS API issues

\section{Conclusions}\label{conclusions}

\section*{Acknowledgements}\label{acknowledgements}
\addcontentsline{toc}{section}{Acknowledgements}

The authors are grateful for the assistance and support of the following
people and institutions without which this work would not have been
possible.

\textbf{FIXME!}

\textbf{PAST ICE DEVS} \textbf{NEAMS Colleagues} \textbf{CASL
colleagues} \textbf{Eclipse colleagues and the community}

This work has been supported by the US Department of Energy, Offices of
Nuclear Energy (DOE-NE) and Energy Efficiency and Renewable Energy
(DOE-EERE), and by the ORNL Undergraduate Research Participation
Program, which is sponsored by ORNL and administered jointly by ORNL and
the Oak Ridge Institute for Science and Education (ORISE). This work was
also supported in part by the Oak Ridge National Laboratory Director's
Research and Development Fund. ORNL is managed by UT-Battelle, LLC, for
the US Department of Energy under contract no. DE-AC05-00OR22725. ORISE
is managed by Oak Ridge Associated Universities for the US Department of
Energy under contract no. DE-AC05-00OR22750.

\section*{Required Metadata}\label{required-metadata}
\addcontentsline{toc}{section}{Required Metadata}

\section*{Current code version}\label{current-code-version}
\addcontentsline{toc}{section}{Current code version}

--------- \textbar{}
:-----------------------------------------------------------------:
\textbar{} :-------------------------------------------------------- :
\textbar{}\\C1 \textbar{} Current Code Version \textbar{} `next'
\textbar{}\\C2 \textbar{} Permanent link to code/repository used for
this code version \textbar{} https://github.com/eclipse/ice/tree/next
\textbar{}\\C3 \textbar{} Legal Code License \textbar{} Eclipse Public
License 1.0 \textbar{}\\C4 \textbar{} Code versioning system use
\textbar{} Git \textbar{}\\C5 \textbar{} Software code languages, tools,
and services used \textbar{} Java, OSGi, Eclipse Rich Client Platform,
and Maven \textbar{}\\C6 \textbar{} Compilation requirements, operating
environments \& dependencies \textbar{} Java 1.8 or greater, Maven, and
an internet connection for dependencies \textbar{}\\C7 \textbar{} If
available Link to developer documentation/manual \textbar{}
https://wiki.eclipse.org/ICE \textbar{}\\C8 \textbar{} Support email for
questions \textbar{} ice-dev@eclipse.org \textbar{}

\section*{Current executable software
version}\label{current-executable-software-version}
\addcontentsline{toc}{section}{Current executable software version}

--------- \textbar{}
:-----------------------------------------------------------------:
\textbar{} :-------------------------------------------------------- :
\textbar{}\\S1 \textbar{} Current Code Version \textbar{} 2.2.1
\textbar{}\\S2 \textbar{} Permanent link to executables of this version
\textbar{}
http://www.eclipse.org/downloads/download.php?file=/ice/builds/2.2.1/
\textbar{}\\S3 \textbar{} Legal Software License \textbar{} Eclipse
Public License 1.0 \textbar{}\\S4 \textbar{} Computing
platforms/Operating Systems \textbar{} Windows (32/64-bit), Mac OS/X,
Linux (32/64-bit) \textbar{}\\S5 \textbar{} Installation requirements \&
dependencies \textbar{} Java 1.8 or greater \textbar{}\\S6 \textbar{} If
available, link to user manual - if formally published include a
reference to the publication in the reference list \textbar{}
https://wiki.eclipse.org/ICE \textbar{}\\S7 \textbar{} Support email for
questions \textbar{} ice-users@eclipse.org \textbar{}

\section{References}\label{references}

Yu, Jia, and Rajkumar Buyya. ``A Taxonomy of Workflow Management Systems
for Grid Computing.'' Journal of Grid Computing 3.3--4 (2005): 171--200.
link.springer.com. Web.

Billings, Jay J. et al. ``Designing a Component-Based Architecture for
the Modeling and Simulation of Nuclear Fuels and Reactors.'' Proceedings
of the 2009 Workshop on Component-Based High Performance Computing. New
York, NY, USA: ACM, 2009. 6:1--6:4. ACM Digital Library. Web. 25 May
2016. CBHPC '09.

Pizzi, Giovanni et al. ``AiiDA: Automated Interactive Infrastructure and
Database for Computational Science.'' Computational Materials Science
111 (2016): 218--230. ScienceDirect. Web.

The Nek5000 Team. ``Nek5000 \textbar{} A Spectral Element Code for
CFD.'' Nek5000 Website. 22 Oct. 2014. Web. 29 Dec. 2016.

McAffer, Jeff, Paul VanderLei, and Simon Archer. OSGi and Equinox.
Boston: Addison-Wesley, 2010. Print.

Gaston, Derek et al. ``MOOSE: A Parallel Computational Framework for
Coupled Systems of Nonlinear Equations.'' Nuclear Engineering and Design
239.10 (2009): n. pag. Print.

Fielding, Roy Thomas. ``Architectural Styles and the Design of
Network-Based Software Architectures.'' University of California,
Irvine, 2000. Print.

Burke, Bill. RESTful Java with JAX-RS. California: O-Reilly, 2010.
Print.

\section{Footnotes}\label{footnotes}

1: The authors have identified many different combinations of these
activities that define workflow ``classes.'' When possible, every effort
is made to give the classes funny names such as ``The Re-Run'' or ``The
Graduate Student.''

2: An upcoming update to the API will see the formal introduction of
IWorkflow, IWorkflowTask, and IWorkflowEngine interfaces to bring ICE's
API language closer to other systems. This must be done with care to
satisfy backwards compatibility requirements on the present API.
